\documentclass{article}

%% Defines my own fxns

\renewcommand{\th}{\textsuperscript{th}\xspace}
\newcommand{\nd}{\textsuperscript{nd}\xspace}
\newcommand{\st}{\textsuperscript{st}\xspace}
\newcommand{\rd}{\textsuperscript{rd}\xspace}
\newcommand{\sq}{\textsuperscript{2}\xspace}

%% end fxns

\pagestyle{headings}

\usepackage[round]{natbib}

%% Sweave Package for incorporating R Code

\usepackage{Sweave}  

\DefineVerbatimEnvironment{Sinput}{Verbatim}{fontsize=\small,fontshape=sl}
\DefineVerbatimEnvironment{Soutput}{Verbatim}{fontsize=\small}
\DefineVerbatimEnvironment{Scode}{Verbatim}{fontsize=\small,fontshape=sl}

\bibliographystyle{plainnat}

\begin{document}

\title{Trading with the \texttt{portfolio} package}
\author{by Jeff Enos, Daniel Gerlanc, and David Kane}

%%\VignetteIndexEntry{Using the tradelist class}
%%\VignetteDepends{portfolio}

\maketitle


\setcounter{secnumdepth}{3}



\begin{abstract}
\label{abstract}

Given a set of current holdings and a target portfolio, that is, a set
of desirable holdings to which we would be willing to switch if
trading were free, and that our reasons for trading can be captured
with one or more rank orderings, the \texttt{portfolio} package
provides a way to use multiple measures of desirability to determine
which trades or portions of trades to do.

\end{abstract}

\section{Introduction}

What should we trade now? This question is much more difficult than it
might first appear, and yet thousands of individuals and firms
controlling trillions of dollars must answer it each day. Consider a
simple example.

Imagine that the investment universe is restricted to 10 securities
and that our portfolio must hold 5 equal-weighted long positions.  At
any given point in time, we will hold one of those portfolios. The
simplest possible ``trade'' is to do nothing, keeping the same
portfolio in the next period that we hold in the current one. A period
can be 5 minutes or 5 months or any length of time. The next simplest
trade is a single position swap. Trade one of our 5 current holdings
for one of the 5 securities not in the portfolio. There are 25 such
trades.  Continuing up the complexity scale, there are 100 trades in
which we replace 2 securities in the portfolio with 2 securities not
in the portfolio. Considering all sets of possible trades, there are
252 options (including the option of no trading), which is equal to
the total number of possible portfolios, $10 \choose 5$.

In a world of perfect information, we would know the future
returns for each of the 10 securities in the universe. Given this
information, and some preferences with regard to risk and return, we
could examine all 252 options and determine which was best.
Unfortunately, in a real world example with thousands of securities in
the universe and possibly hundreds in the portfolio, there is no way
to forecast results for each possible combination.

These difficulties are exacerbated by other complications.

\begin{itemize}

\item{\bf{Ranking Trades}}: We may have multiple criteria for ranking
  the trades.  Some criteria may be more appropriate for ranking
  certain types of trades under specific circumstances.  In the case
  where we have a large number of criteria, how do we choose the most
  appropriate criterion for each trade?

\item{\bf{Liquidity}}: Even if it were simple to determine the ideal
  portfolio, it may be difficult to get there. Imagine that moving to
  the ideal portfolio requires that we trade one million shares of
  IBM; however, IBM typically trades 100,000 shares a day. How are we
  going to buy all the necessary shares in one day? Even if we bought
  the entire day's volume (an impossibility) it would take we ten days
  to get the entire position.

\item{\bf{Trade Costs}}: Trading is not free so we will want to do less of
  it in the real world than we might care to do in the theoretical
  world. Basic trading costs (including commissions and spread) tend
  to enter the calculation linearly. Trade twice as much and we pay
  twice the costs. 

\item{\bf{Turnover}}: Turnover is the flip-side of holding period.  In an
  ideal world, holding period would be endogenous. We would select
  the holding period which maximised the risk-adjusted return of the
  portfolio. But, in the real world, almost all portfolios have
  targeted holding periods that should be adhered to. We are only
  allowed a certain amount of turnover.

\item{\bf{Price Impact}}: Although commission/spread may be linear in
  trade volume, price impact is not.  We are a participant in the
  market, and every time we trade we affect the price.  Price impact
  is generally small if we trade a modest portion, say 10\%, of
  volume.  But if we trade more, then the price will move against
  we.  Over some range, price impact increases more than linearly.
 
\end{itemize}

None of these problems is impossible to overcome, but all of them
conspire to make a general solution to the trading problem extremely
difficult. Therefore, we simplify.

\label{simplifying assumption}

The \texttt{portfolio} package makes three major simplifying
assumptions.  The first is that we have created a ``target'' or
``ideal'' portfolio, a set of positions that is desirable and to which
we would be willing to switch if trading were free. This assumption is
implausible but it does serve to make the problem tractable. If we
only consider trades which move us closer to the target portfolio, it
is much easier to handle the other difficulties associated with
turnover, liquidity and the like. Instead of looking at all possible
buys, for example, we only need to analyse buys for securities in
which the target portfolio has more shares than current holdings.  The
second simplifying assumption is that different criteria for trading
can be captured with a rank ordering.  We discard the information used
to create the ranks.  The third simplifying assumption is that no one
type of trade is intrinsically better than another type of trade.  All
things equal, buys, sells, covers, and shorts are equally good.

We divide the document into three sections.  First, we create a trade
list.  Second, we create a trade list in R.  Third, we create a
long-short trade list in R.

\section{A simple example}



Assume that we already have a small portfolio consisting of positions
in various equities.  We have been given an additional \$1,000 to
invest in the portfolio, and we must invest this \$1,000 over the
course of one trading day.  This is not a realistic scenario, but
having a set amount of time in which to trade will simplify our
example.  Throughout the document, we will refer to our present
holdings as the ``current'' portfolio.  The ``target portfolio'' is an
ideal set of holdings that we would immediately switch to if trading
were free as per the first simplifying assumption.  Note that in this
simple example the only trades we will be considering are buys.  In
subsequent we will introduce more complexity in the form of buys,
sells, covers, and shorts.

\subsection{Current and target holdings}

Our current portfolio consists of shares of
3 companies, IBM (International Business
Machines), GM (General Motors) and EBAY (EBay).

\begin{Schunk}
\begin{Soutput}
     shares price
IBM     100    10
GM      100    30
EBAY     75   120
\end{Soutput}
\end{Schunk}

The \texttt{shares} column expresses how many shares of each stock are
in the portfolio, and the \texttt{price} column expresses the most
recent price of that equity.\footnote{For simplicity, we use US
dollars.}  The market value of the current portfolio can be calculated
by summing the products of the shares and prices.

As per the simplifying assumption, we provide a target portfolio.

\begin{Schunk}
\begin{Soutput}
     shares price
GOOG     50    20
EBAY     75   120
IBM     100    10
GM      200    30
SCHW    100    50
MSFT    100    60
\end{Soutput}
\end{Schunk}

We would like to buy more shares of GM and take positions in SCHW
(Charles Schwab Inc.), MSFT (Microsoft), and GOOG (Google).  The
market value of the target portfolio is
\$28,000.
 
\subsection{Portfolio difference}

The portfolio difference may be understood as the trades that would
change our current holdings into our ideal holdings. If trading were
free, we would immediately complete all these trades and reach our
target portfolio.  Alas, trading is not free, and we will most likely
not complete all the orders in one day.  Some of them probably require
that we purchase a large portion of the daily trading volume (over
10\%), at which point the trade may become significantly less
desirable.

From the portfolio difference, we determine our \emph{candidate
trades}.

\begin{description}

\item{\bf{candidate trades}}: The complete set of trades we would have
to make to trade from our current portfolio to the target portfolio.
If trading were free, we would make all of these trades right now.

\end{description}

Below, we list the candidate trades.

\begin{Schunk}
\begin{Soutput}
     side shares   mv
GM      B    100 3000
GOOG    B     50 1000
MSFT    B    100 6000
SCHW    B    100 5000
\end{Soutput}
\end{Schunk}

The \texttt{side} column expresses what type of trade we will be
making.\protect\footnote{In later examples, S will represent a sell, X
will represent a short and C will represent a cover.}  All the
candidate trades are buys so the \texttt{side} column only contains
\texttt{B}.  The \texttt{shares} column expresses the number of shares
of each stock we must buy to reach the target portfolio.  The
\texttt{mv} column expresses the effect that candidate will have on
the value of the portfolio.  Buys, which increase the value of our
portfolio, have a positive value.  Sells, which decrease the value of
the portfolio, have a negative value.

As the market value of the target portfolio
(\$28,000) is
greater than the market value of the original portfolio
(\$13,000),
we would have to invest an additional
\$15,000
to trade from our current portfolio to our target portfolio.  However,
we only have \$1,000 with which we may buy additional shares.
Therefore, we have to decide which subset of the candidate trades we
will make.

\subsection{Expressing preferences amongst trades}

Part of our simplifying assumption is that we would instantly switch
to the target portfolio if trading were free.  This implies that all of
the candidate trades are desirable.  However, they are not all equally
desirable.  Some trades are better than others.  We want to determine
which candidate trades or subsets of the candidate trades yield the
most utility on the margins.

If we had unlimited funds or could freely trade between our current
and target portfolios, we would not have to express preferences
amongst trades.  However, in the real world, we must decide, given a
set of possible trades, which trades we should make first.  One way to
do this involves assigning each trade a value of overall desirability,
\emph{alpha}.  When we associate trades with alpha, we say that we
sort by alpha or use alpha as a \emph{sort.}  

\begin{description}

\item{\bf{alpha}}: a value, most likely generated by some sort of
  quantitative model, which expresses the relative quality of the
  candidate trades.  Serves to determine from which candidate trade we
  would derive the most utility.

\item{\bf{sort}}: a set of values associated with a set of positions.
Higher values suggest positive future performance and negative values
suggest poor future performance.  Based on a sort we can determine
what type of trade is most desirable for each position.  Therefore, we
want to buy or cover positions with positive sort value and sell or
short positions with negative sort values.

\end{description}

Like portfolio construction, alpha generation is beyond the scope of
this document; we provide alpha values and will be using alpha as a
sort.  We associate good buys with greater alpha values.  In the table
below, the candidate at the head of the data frame has the highest
alpha value and is therefore the most desirable trade.

\begin{Schunk}
\begin{Soutput}
     side alpha
MSFT    B   1.5
SCHW    B   1.2
GOOG    B   1.1
GM      B   1.0
\end{Soutput}
\end{Schunk}

Based on the alpha values, MSFT is the best trade, SCHW is the second
best trade, and GM is the worst trade with an alpha value of
1.

\subsection{Preliminary ranks}

We determine which trades are most desirable by generating an overall
measure of desirability for each trade.  The first step in generating
this value involves creating a \emph{rank ordering} of the trades for each
sort we have created.  A definition of this term follows:

\begin{description}

  \item{\bf{rank ordering}}: a linear, relational ordering of the
  candidates, where each candidate is assigned a rank from the set $1, 2,
  3\dots{}n$ where $n$ is the number of candidate trades.  Trade 1
  provides the greatest  utility and trade $n$ provides the
  least  utility.  In creating a rank ordering we discard
  cardinal information such as alpha or one-day return and replace it
  with a whole number ranking.


\end{description}

We rank and order the candidates by alpha below:

\begin{Schunk}
\begin{Soutput}
     rank side alpha shares   mv
MSFT    1    B   1.5    100 6000
SCHW    2    B   1.2    100 5000
GOOG    3    B   1.1     50 1000
GM      4    B   1.0    100 3000
\end{Soutput}
\end{Schunk}

Notice that a rank ordering abstracts the underlying data.  Whereas
alpha provides an absolute measure of desirability, rank expresses the
relative desirability amongst trades.  We say that we lose
\emph{cardinal information} when we use ranks.

\begin{description}

\item{\bf{cardinal information}}: The values used to create a rank
  ordering.  The creation of ranks abstracts these values and replaces
  them with an ordering that reflects the value of an element relative
  to other elements in the rank ordering.

\end{description}

In some cases we may want to use more than one measure of
desirability.  We may have more than one source of cardinal
information.  Imagine that we want to use both alpha and one-day
return as sorts.  If we believe in one day reversal, we would assign
higher ranks to both orders to sell stocks with positive one-day
returns and to orders to buy stocks with negative one-day returns.
However, we associate more desirable buys with greater sort values.
To account for this, we have taken the inverse of all one-day return
values.  Therefore, if the return one-day return for GM appears to be
$0.10$, it should really be $-0.10$.  The table below and to the left
shows the one-day return values after we have taken their inverse.
The table to the right contains that actual one-day return values.

\begin{verbatim}

     rank side ret.1.d                rank side ret.1.d
GM      1    B    0.10                GM      1    B   -0.10
GOOG    2    B    0.01                GOOG    2    B   -0.01
MSFT    3    B   -0.01                MSFT    3    B    0.01
SCHW    4    B   -0.02                SCHW    4    B    0.02

\end{verbatim}


GM has the highest rank according to one-day return
because it has the most negative return of all the buys, but for the
\texttt{portfolio} package to properly process our one-day reversal
sort, we must take the inverse of all the one-day return values.

\subsubsection{The problem of multiple sorting criteria}

When we combine the sorts in a single data frame, it is not clear
which sort values we should use. If we order by alpha we get the
following set of ranks:

\begin{Schunk}
\begin{Soutput}
     rank alpha ret.1.d
MSFT    1   1.5   -0.02
SCHW    2   1.2   -0.01
GOOG    3   1.1    0.01
GM      4   1.0    0.10
\end{Soutput}
\end{Schunk}

Ordering by one-day return yields another ordering:

\begin{Schunk}
\begin{Soutput}
     rank alpha ret.1.d
GM      1   1.0    0.10
GOOG    2   1.1    0.01
SCHW    3   1.2   -0.01
MSFT    4   1.5   -0.02
\end{Soutput}
\end{Schunk}

When we use multiple sorts, there is no obvious way in which we would
order the trades by desirability.  When sorting by alpha,
MSFT is the most desirable trade, and when
sorting by one-day return, GM is the most
desirable trade.  We cannot easily compare or combine the two sorts
because we do not know what the exact relationship is between one-day
reversal and alpha.  Neither sort is even on the same numeric scale.
Should we alternate between using values from alpha and one-day
return?  How would we decide how often to alternate between the sorts?
The way in which we express preferences amongst trades in the
\texttt{portfolio} package represents our answer to these questions.
In sections \ref{words weighting sorts} through \ref{generating
synthetic ranks} we discuss our method for ranking trades when we
multiple measures of desirability.  In section \ref{sorting theory} we
will discuss the reasoning behind our method of ordering trades.


%% As per our second simplifying assumption, we lose a certain amount of
%% data when we replace cardinal information with ranks.  In doing this
%% the \texttt{portfolio} package makes the important assumption, that
%% all our different criteria for trading can be captured with a rank
%% ordering.  Anyone who uses the package should be aware of this
%% assumption.  This assumption and the assumption that the user provides
%% a target portfolio are the most significant assumptions we make.

%% However, if we did not only consider ranks, it would be difficult to
%% compare sorts.  We would have to force the user to provide a function
%% that expresses the relationship between the sorts.  Writing such a
%% function is hard, especially if the we use multiple sorts.  To
%% facilitate the comparison of sorts, we introduce a weighting scheme.

\subsection{Weighting sorts}
\label{words weighting sorts}

As sorts express preferences amongst trades, weights express
preferences amongst sorts.  By assigning each sort a weight, we
express how important that sort is relative to other sorts.  To
illustrate some weighting examples, let's consider the scenario in
which we have assigned a weight of 1 to both the alpha and one-day
return sorts.  By assigning the same weight to both sorts we assert
that they are equally important.  Assigning a weight directly affects
the sort rankings by causing them to be divided by the weight.
However, we have assigned both the sorts a weight of 1 so the ranks
remain the same.

\begin{Schunk}
\begin{Soutput}
     rank side alpha shares   mv
MSFT    1    B   1.5    100 6000
SCHW    2    B   1.2    100 5000
GOOG    3    B   1.1     50 1000
GM      4    B   1.0    100 3000
\end{Soutput}
\end{Schunk}

However, the ranks for one-day return remain the same because one-day
return has a weight of 1.

\begin{Schunk}
\begin{Soutput}
     rank side ret.1.d shares   mv
GM      1    B    0.10    100 3000
MSFT    2    B   -0.02    100 6000
SCHW    3    B   -0.01    100 5000
GOOG    4    B    0.01     50 1000
\end{Soutput}
\end{Schunk}

Having divided the original, \emph{raw ranks} by weight, we now have
\emph{weighted ranks}.

\begin{description}

\item{\bf{raw ranks}}: the original, linearly spaced ranks, built on
  the scale $1, 2, 3\dots{}n$

\item{\bf{weighted ranks}}: the raw ranks divided by the weights of
  the sorts.\protect\footnote{Note that we are using the term ranks in
  several different contexts, a possibly abusive notation.}

\end{description}

We now have two ranks associated with each candidate, one from the
alpha sort and another from the one-day return sort.  To illustrate
that we have duplicate ranks for each sort, we bind the
equally-weighted alpha and one-day return sorts to form a single data
frame.

\begin{Schunk}
\begin{Soutput}
             rank    sort side shares   mv
MSFT.alpha      1   alpha    B    100 6000
GM.ret.1.d      1 ret.1.d    B    100 3000
SCHW.alpha      2   alpha    B    100 5000
MSFT.ret.1.d    2 ret.1.d    B    100 6000
GOOG.alpha      3   alpha    B     50 1000
SCHW.ret.1.d    3 ret.1.d    B    100 5000
GM.alpha        4   alpha    B    100 3000
GOOG.ret.1.d    4 ret.1.d    B     50 1000
\end{Soutput}
\end{Schunk}

The row names express the equity ticket symbols, and the string
following the period is the name of the sort that generated this rank.
At each rank there are two candidates, one of which has been
associated with a rank from alpha and the other which has been
associated with a rank from one-day return.  In cases such as this
where we have equally weighted sorts there will be a candidate trade
from each sort at every rank.

If we use $n$ sorts, we will have $n$ ranks associated with each
candidate.  We only want one rank associated with each candidate.  So
that each candidate only has one rank associated with it, we assign
each rank the best rank generated for it by any sort.  We have done
this in the data frame below.


\begin{Schunk}
\begin{Soutput}
     rank shares   mv
GM      1    100 3000
MSFT    1    100 6000
SCHW    2    100 5000
GOOG    3     50 1000
\end{Soutput}
\end{Schunk}

Both GM and MSFT have been assigned a rank of one.  This occurs
because MSFT has been ranked 1 by the alpha sort and GM has been
ranked 1 by the one-day return sort.  SCHW has been ranked 2 by the
alpha sort and GOOG has been ranked 3 by the alpha sort.

When we equally weight the sorts we are equally likely to use ranks
from one either sort.  This behaviour is logical because
assigning sorts equal weights suggests that they are equally
important.  However, the sorts may not always be equally important.
In the next example we use a weighting scheme that causes us to use
one sort to the exclusion of the other.

Let's say that we do not want to consider one-day return.  To ignore
all of the one-day return values, we make alpha 10 times more
important than one-day return.  Therefore, we will consider 10 ranks
from alpha for every one rank from one-day return.  As there are only
4 candidate trades, we will choose the
rankings in alpha over the ranks in the one-day return sort.

\begin{Schunk}
\begin{Soutput}
             rank side shares   mv
MSFT.alpha    0.1    B    100 6000
SCHW.alpha    0.2    B    100 5000
GOOG.alpha    0.3    B     50 1000
GM.alpha      0.4    B    100 3000
GM.ret.1.d    1.0    B    100 3000
MSFT.ret.1.d  2.0    B    100 6000
SCHW.ret.1.d  3.0    B    100 5000
GOOG.ret.1.d  4.0    B     50 1000
\end{Soutput}
\end{Schunk}

Creating this unbalanced weighting causes us to stack the alpha ranks
on top of the one-day return ranks.  Since we always assign the lowest
rank from all trades to a sort, we will consider the alpha ranks
before any of the one-day return ranks.

\begin{Schunk}
\begin{Soutput}
     rank  sort shares   mv
MSFT  0.1 alpha    100 6000
SCHW  0.2 alpha    100 5000
GOOG  0.3 alpha     50 1000
GM    0.4 alpha    100 3000
\end{Soutput}
\end{Schunk}

Making the alpha 10 times as important as the one-day return sort
causes us to only use ranks from the alpha sort.  We do not even
consider the number 1 ranked one-day return trade until we examine all
the alpha sorts ranked in the top ten.  As we only have 4 candidate
trades, we do not consider any trades from one-day return.

The last weighting we will consider falls somewhere in between the
previous two.  

\begin{Schunk}
\begin{Soutput}
             rank side shares   mv
MSFT.alpha   0.67    B    100 6000
GM.ret.1.d   1.00    B    100 3000
SCHW.alpha   1.33    B    100 5000
GOOG.alpha   2.00    B     50 1000
MSFT.ret.1.d 2.00    B    100 6000
GM.alpha     2.67    B    100 3000
SCHW.ret.1.d 3.00    B    100 5000
GOOG.ret.1.d 4.00    B     50 1000
\end{Soutput}
\end{Schunk}

To weight the alpha sort by an additional 50\%, we divide all the
ranks in alpha by 0.5.  This causes us to consider 3 ranks from the
alpha sort for every 2 rank from the one-day return sort.

\begin{Schunk}
\begin{Soutput}
     rank shares   mv
MSFT 0.67    100 6000
GM   1.00    100 3000
SCHW 1.33    100 5000
GOOG 2.00     50 1000
\end{Soutput}
\end{Schunk}

We use three of the ranks from the alpha sort and one
rank from the one-day return sort.  This is the weighting scheme that
we will use in the rest of the example.

We use three of the ranks from the alpha sort and one
rank from the one-day return sort.  This is the weighting scheme that
we use in the rest of the example.

To review, the ranking process has four steps.  First, we ranked each
trade according to both alpha and one-day return to generate raw
ranks.  Second, we weight these ranks.  Third, we combined the alpha
and one-day return ranks.  Fourth, we eliminated duplicates by
associating each trade with the lowest rank assigned to it by either
alpha or one-day return.  We call these ranks preliminary ranks
because they are not the final values we use to determine the
desirability of each trade.  Nonetheless, we must generate preliminary
ranks before we can arrive at final ranks, the calculation of which we
describe in the next section.

\subsection{Generating synthetic ranks}
\label{generating synthetic ranks}

If trade 1 is $X$ better than the trade 2, then is trade 99 $X$ better
than trade 100?  Most portfolio managers would argue that the
difference in utility between trade 1 and trade 2 is greater than the
difference in utility between trade 99 and trade 100.  However, with
raw ranks, we make no assertion of how much better one trade is than
another trade.  To express the tendency for us to derive more utility
from the most highly ranked trades, we synthesise yet another set of
values from the weighted ranks.  We call these values synthetic
ranks.\protect\footnote{We are abusing the term ``ranks'' by using it
in several different contexts.}

\begin{description}

\item{\bf{synthetic ranks}}: values generated by mapping the weighted
ranks to a truncated normal distribution ($> 85^{th}$ percentile on
$N(0,1)$).

\end{description}

First, we re-rank the weighted ranks:

\begin{Schunk}
\begin{Soutput}
     rank shares   mv
MSFT    1    100 6000
GM      2    100 3000
SCHW    3    100 5000
GOOG    4     50 1000
\end{Soutput}
\end{Schunk}

Next, we evenly distribute the ranks on the interval interval
$[0.85,1)$ such that the best ranked trades are closest to 1 and the
worst ranked trades are closest to $0.85$:

\begin{Schunk}
\begin{Soutput}
     rank shares   mv rank.s
MSFT    1    100 6000   0.97
GM      2    100 3000   0.94
SCHW    3    100 5000   0.91
GOOG    4     50 1000   0.88
\end{Soutput}
\end{Schunk}

We list the scaled ranks in \texttt{rank.s}.  Next, we map to the a
truncated normal distribution.\protect\footnote{$> 85^{th}$ percentile
of $N(0,1)$}


\begin{Schunk}
\begin{Soutput}
     rank rank.s rank.t shares   mv
MSFT    1   0.97    1.9    100 6000
GM      2   0.94    1.6    100 3000
SCHW    3   0.91    1.3    100 5000
GOOG    4   0.88    1.2     50 1000
\end{Soutput}
\end{Schunk}

The \texttt{rank.t} column lists the ranks mapped to the truncated
normal distribution.  GM has the best rank and
SCHW, has the lowest rank.  We might
expect to see a \texttt{rank.t} of approximately 3.5 for the best
ranked trade, but because we only have 4 candidates
and the scaled values are evenly spaced on the interval $[0.85,1)$,
the normalised value of the best ranked trade is not as great as
it would be if we had 100 trades.

Recall that synthetic ranks express the tendency for there to be
greater differences in utility between adjacent, highly ranked trades
($1,2,3\dots{}$) than between adjacent, poorly ranked trades: \\*

\begin{table}[!htbp]
  \begin{tabular}[c]{|rr|rr|rr|}
    \hline
    rank & $\Delta$ & $N(0,1)$ & $\Delta$ & $> 85^{th}$ of $(0,1)$ & $\Delta$ \\
      \hline
      1   & 1  &  3.50 & 1.17 & 3.50 & 0.53 \\
      2   & 1  &  2.32 & 0.27 & 2.96 & 0.21 \\
      3   & 1  &  2.05 & 0.17 & 2.74 & 0.13 \\
      4   & 1  &  1.88 & 0.13 & 2.61 & 0.10 \\
      5   & 1  &  1.75 & -    & 2.51 & -    \\
      .   & .  &     . &    . &    . &    . \\
      .   & .  &     . &    . &    . &    . \\
      48  & 1  &  0.05 & 0.03 & 1.46 & 0.01 \\
      49  & 1  &  0.02 & 0.02 & 1.45 & 0.01 \\
      50  & 1  &  0.00 & 0.02 & 1.44 & 0.01 \\
      51  & 1  & -0.02 & 0.02 & 1.43 & 0.01 \\
      52  & 1  & -0.05 & -    & 1.41 & -    \\
      .   & .  &     . &    . &    . &    . \\
      .   & .  &     . &    . &    . &    . \\
      96  & 1  & -1.64 & 0.11 & 1.06 & 0.00 \\
      97  & 1  & -1.75 & 0.13 & 1.06 & 0.00 \\
      98  & 1  & -1.88 & 0.17 & 1.06 & 0.00 \\
      99  & 1  & -2.05 & 0.27 & 1.06 & 0.00 \\
      100 & -  & -2.32 & -    & -    & NA   \\
      \hline
  \end{tabular}
  \caption[Synthetic rank distributions]{Creating synthetic ranks
    using a linear distribution, a normal distribution, and a
    truncated normal distribution.  Delta columns express the
    difference in utility between adjacent trades.\label{distribution
    table}}
\end{table}

Table \ref{distribution table} expresses the differences amongst
distributions we might use to rank 100 trades.  The \texttt{rank}
column contains the raw ranks for the 5 best trades, the 5
middle-ranked trades, and the 5 worst trades.  In this example the
ranks on $[1,100]$ are spaced at intervals of one.  The alpha
difference between every trade is the same.  If we use raw rank as a
measure of alpha, we derive the same utility from every trade.  Trade
1 is one better than trade 2, and trade 99 is one better than trade
100.

The normal distribution column $(N(0,1))$ expresses what happens when
we normalise the raw ranks.  The normal distribution correctly
expresses our belief that there is a large  difference in
alpha between the best ranked trades.  However, use of the normal
distribution would incorrectly suggest that there are similarly large
alpha differences between the worst trades.  We get these results when
using the normal distribution because the best and worst ranked trades
form the tails of the distribution.  We do not want large 
differences in alpha amongst the worst rank trades.  The 
alpha differences decrease until we reach trade 50, then increase
again as we move towards the other tail of the distribution.  We want
alpha to remain the same on the margin past the 50th trade.

To address the problems of a normal and linear distribution, we use a
truncated normal distribution, $> 85^{th} \% of N(0,1)$.  In the
right\-most delta $(\Delta)$ column, the  alpha differences
between the best ranked trades is over 50 times greater than the
 alpha differences between the worst ranked trades.  Every
trade ranked worse than 50 has a similar  alpha difference.
Although the subset $[0.85,1)$ is slightly arbitrary, (we could have
set the lower extreme to be 0.84, 0.85, or another similar value) it
serves our purpose of expressing large differences in  alpha
and where we find the best buys, at one tail, and small differences in
 alpha amongst the worst buys.

Recall the steps we have taken towards in generating our final measure
of rank, synthetic rank.  First, we converted the sort values to raw
ranks.  Second, we converted the raw ranks to weighted ranks.  Third,
we scaled the weighted ranks to $[0.85,1)$ to generate scaled weights.
Lastly, we mapped the scaled weights to the truncated normal
distribution.  By only using the $85^{th}$ percentile and above, we
express our belief that the differences in alpha between the best
ranked trades is much greater than the differences in alpha between
the worst ranked trades.

If the costs associated with trading any stock, all things being
equal, were the same, we would not care about the difference in
utility between any trades.  We would move down the trade list from
best to worst until we match the allotted turnover.  However, our
trading influences prices and may reduce the desirability of a trade.


\subsection{Chunks, synthetic rank, and trade-cost adjustment}
\label{Chunks, synthetic rank, and trade-cost adjustment}

We want to know at what point the cost of trading an equity exceeds
the utility of trading that equity.  In the \texttt{portfolio}
package, we use synthetic rank to represent utility.  Determining the
cost of purchasing an additional share is impossible if our smallest
trading unit is an entire order so we break each order into
\emph{chunks}.

\begin{description}

\item{\bf{chunk}}: A portion of a candidate trade.

\end{description}

We break candidate trades into chunks by market value.  Each chunk has
a market value of approximately \$2000:

\begin{Schunk}
\begin{Soutput}
       side shares   mv alpha ret.1.d rank.t chunk.shares chunk.mv
GM.1      B    100 3000   1.0    0.10    1.6           67     2010
GM.2      B    100 3000   1.0    0.10    1.6           33      990
GOOG.1    B     50 1000   1.1    0.01    1.2           50     1000
MSFT.1    B    100 6000   1.5   -0.02    1.9           33     1980
MSFT.2    B    100 6000   1.5   -0.02    1.9           33     1980
MSFT.3    B    100 6000   1.5   -0.02    1.9           33     1980
MSFT.4    B    100 6000   1.5   -0.02    1.9            1       60
SCHW.1    B    100 5000   1.2   -0.01    1.3           40     2000
SCHW.2    B    100 5000   1.2   -0.01    1.3           40     2000
SCHW.3    B    100 5000   1.2   -0.01    1.3           20     1000
\end{Soutput}
\end{Schunk}

The candidate trades are broken into 10 chunks.
The number following the period in the row name expresses the chunk
number for that particular equity.  The \texttt{chunks.mv} column
expresses the market value of each chunk. The \texttt{chunk.shares}
column expresses how many shares each chunk consists of.

\subsubsection{Trade-cost adjustment of individual chunks}

As we trade greater percentages of the average daily volume, the price
of the equity will increase.  To reflect this phenomenon, we penalise
the synthetic ranks of the chunk as we trade greater percentages of
the daily volume.  We call this penalty \emph{trade-cost adjustment}.

\begin{description}

\item{\bf{trade-cost adjustment}}: Lowering a chunk's rank because of
trading volume.

\end{description}

To fix this idea, let's first examine the daily volumes of our
candidate trades.\protect\footnote{The \texttt{volume} column
represents some measure of past trading volume such as the average
trading volume over the last 30 days.  A daily measure of
\texttt{volume} is not required; we would use whatever measure is
natural for the frequency with which we trade.}

\begin{Schunk}
\begin{Soutput}
     rank.t volume shares
GM      1.6    100    100
GOOG    1.2   2200     50
MSFT    1.9   2600    100
SCHW    1.3   2500    100
\end{Soutput}
\end{Schunk}

The trades we want to make for both MSFT, SCHW, and GOOG involve less
than 3\% of the daily trading volume.  However, we must trade 100\% of
the daily trading volume of GM.  We would probably not be able to
purchase all these shares in one day, and even if we could, we would
affect prices significantly.  Moving into the position over several
days would be better.

We use a trade-cost adjustment function to express how increasing
trade costs reduce the desirability of candidate trades.  To better
approximate utility, we penalise synthetic ranks at the chunks level.
Doing this allows us to better determine at which point the cost of
trading an additional chunk is greater than the utility derived by
trading an additional chunk.  We perform trade-cost adjustment on the
chunks by keeping track of what percentage of the daily volume we have
traded with each additional chunk.  The first chunk to cross the
threshold of 15\% of the daily trading volume is penalised by a fixed
amount.  All subsequent chunks are penalised by that amount, and any
further chunks that pass 30\% or 45\% percent of the daily trading
volume receive further penalties.  The chunks of GM, an illiquid
equity, have been trade-cost adjusted.

\begin{Schunk}
\begin{Soutput}
       side   mv alpha ret.1.d rank.t chunk.shares chunk.mv tca.rank
GM.1      B 3000   1.0    0.10    1.6           67     2010      1.6
GM.2      B 3000   1.0    0.10    1.6           33      990     -4.4
GOOG.1    B 1000   1.1    0.01    1.2           50     1000      1.2
MSFT.1    B 6000   1.5   -0.02    1.9           33     1980      1.9
MSFT.2    B 6000   1.5   -0.02    1.9           33     1980      1.9
MSFT.3    B 6000   1.5   -0.02    1.9           33     1980      1.9
\end{Soutput}
\end{Schunk}

The \texttt{tca.rank} column expresses the synthetic rank adjusted for
trade-cost.  As the only candidate for which we want to purchase more
than 15\% of the daily trading volume is GM, it is the only candidate
for which we trade-cost adjust the chunks.  Every chunk of GM beyond
the first has been trade-cost adjusted.  This will cause us to
consider the chunks of other candidate trades before we trade
additional chunks of GM:

\begin{Schunk}
\begin{Soutput}
       side   mv alpha ret.1.d rank.t chunk.shares chunk.mv tca.rank
MSFT.1    B 6000   1.5   -0.02    1.9           33     1980      1.9
MSFT.2    B 6000   1.5   -0.02    1.9           33     1980      1.9
MSFT.3    B 6000   1.5   -0.02    1.9           33     1980      1.9
MSFT.4    B 6000   1.5   -0.02    1.9            1       60      1.9
GM.1      B 3000   1.0    0.10    1.6           67     2010      1.6
SCHW.1    B 5000   1.2   -0.01    1.3           40     2000      1.3
SCHW.2    B 5000   1.2   -0.01    1.3           40     2000      1.3
SCHW.3    B 5000   1.2   -0.01    1.3           20     1000      1.3
GOOG.1    B 1000   1.1    0.01    1.2           50     1000      1.2
GM.2      B 3000   1.0    0.10    1.6           33      990     -4.4
\end{Soutput}
\end{Schunk}

As MSFT is the best ranked candidate and does not receive a trade-cost
penalty, we would trade all the shares of MSFT before considering the
other candidates.\protect\footnote{Assuming that derived turnover is
greater than the market value of all the candidate trades.}  Having
completed all the trades of MSFT, we would consider the first chunk of
GM, the only chunk which has not been trade-cost adjusted.
Subsequently, we would trade all the chunks of SCHW and GOOG, the
candidate trades ranked 3 and 4.  Lastly, we trade the penalised
chunks of GM.

\subsubsection{Synthetic rank and trade-cost adjustment of small portfolios}
\label{Synthetic rank and trade-cost adjustment of small portfolios}
In this example, trade-cost adjustment decreases the desirability of
the chunks of GM in a non-trivial way.  Although GM is ranked 2nd as a
candidate trade, every other candidate trade would be made before we
completed all the chunks of GM.  When we consider such a small number
of trades, we assume that all of the trades are of approximately equal
quality; the  difference in utility between candidate trades
is fairly small.  This occurs because the scaled ranks are evenly
distributed on $[0.85,1)$:

\begin{Schunk}
\begin{Soutput}
     rank shares   mv rank.s
MSFT    1    100 6000   0.97
GM      2    100 3000   0.94
SCHW    3    100 5000   0.91
GOOG    4     50 1000   0.88
\end{Soutput}
\end{Schunk}

  When we only have 4 candidates, none of
the scaled ranks will be very close to $1$, and consequently, none of
the synthetic ranks will fall at the extreme tail
of the normal distribution:

\begin{Schunk}
\begin{Soutput}
     rank shares   mv rank.s rank.t
MSFT    1    100 6000   0.97    1.9
GM      2    100 3000   0.94    1.6
SCHW    3    100 5000   0.91    1.3
GOOG    4     50 1000   0.88    1.2
\end{Soutput}
\end{Schunk}

Consequently, the  difference in utility between candidate
trades will be small when there are few candidate trades.
Heuristically, this seems correct because if we are making very few
trades, we would most likely derive similar utility from any of
them.\protect\footnote{This does not exclude our expressing a
preference amongst the sorts.}  Therefore, it makes sense for us to
trade the other three candidates if the costs associated with trading
GM are non\-trivial.

\subsubsection{Synthetic rank and trade-cost adjustment of large portfolios}
\label{Synthetic rank and trade-cost adjustment of large portfolios}

Moving away from our example for a moment, imagine that we have a
large current and target portfolio, the trade list for which contains
100 candidate trades.  When we have a large portfolio, we tend to
view the differences in  utility between candidates in the
manner we described in section \ref{}.  When evenly distribute the
scaled ranks on the interval $[0.85,1)$, we have more ranks at
the extreme tail:

\begin{Schunk}
\begin{Soutput}
     rank rank.s rank.t
IBM     1   1.00    3.0
GOOG    2   1.00    2.8
GM      3   1.00    2.6
MS      4   0.99    2.5
SCHW    5   0.99    2.4
MSFT   48   0.92    1.4
T      49   0.92    1.4
CVX    50   0.92    1.4
AET    96   0.86    1.1
AMD    97   0.86    1.1
DELL   98   0.85    1.1
EBAY   99   0.85    1.0
HPQ   100   0.85    1.0
\end{Soutput}
\end{Schunk}

The row names express the equity ticker symbols.  \texttt{rank} is the
raw rank. \texttt{rank.s} is the scaled rank, and \texttt{rank.t} is
the synthetic rank.  The best ranked trade
IBM, has a scaled rank value very
close to one, , and
a synthetic rank close to three.  This indicates that the best rank
falls at the tail of the normal distribution.  The worst ranked
candidates not only have low synthetic ranks, but they also have very
small  differences in synthetic rank.  If we trade-cost adjust
one of the poorly ranked candidates we will most likely not trade it
until we have traded all other candidates not penalised by trade cost
adjustment.  On the other hand, we would still trade
IBM,
GOOG, or
GM, even if some of the chunks had
been trade-cost adjusted:

Here we have a subset of the hypothetical chunk table for the 100
candidate example.  For this example, the GOOG candidate has been
broken up into 2 chunks and the IBM candidate has been broken up into
4 chunks.  The ranks of 2$^{nd}$, 3$^{rd}$, and
4$^{th}$ chunks of IBM have been penalised for trade costs.
Therefore, we trade the first chunk of IBM, followed by all the chunks
of GOOG.  Subsequently, we trade the remaining chunks of IBM because
the trade-cost adjusted rank of its chunks is still greater than the
un-penalised synthetic rank of the next most desirable candidate, GM.

Let's quickly review how we generate the final, synthetic ranks.  The
preliminary values from which we draw the raw ranks are the sorts we
define.  In this example, we defined sorts for alpha and one-day
return.  In creating raw ranks abstract away the underlying values
provided by the sorts.  At this point, we still have a different set
of raw ranks for each sorts.  To express preferences amongst the
sorts, we apply weights to the sorts.  This step yields weighted
ranks.  From the sets of weighted ranks, we associate with each
candidate the best weighted rank from any sort.  Next, we scale the
buys to the interval $[0.85,1)$.  This step yields scaled ranks.  From
scaled ranks, we generate synthetic ranks by mapping the scaled ranks
to a truncated normal distribution.  Next, we break the candidates
into chunks and perform trade-cost adjustment as necessary.  This
yields trade-cost adjusted ranks which are the final measure of a
chunks desirability.

\subsection{Sorting theory}
\label{sorting theory}

Chooing the best candidate when we have multiple measures of
desirability is difficult.  Consider the situation where we must
choose ten stocks to trade.

Assuming that we use some type of formula to generate alpha, we might
be able to incorporate our other sorts into the formula for alpha.
Instead of having alpha and one-day return as distinct sorts, we would
only have one sort, alpha, which would also take one-day return into
account.  For this to work, however, we would have to write a function
that accounted for the the ordering of every trade by every sort.
Furthermore, this function would have to take into account our
preference for certain certain sorts over other sorts.  To elaborate
on the difficulty of this creating such a function, let us consider
the situation where we must choose our ten favourite trades, in no
particular order, using the data in the table below.

\begin{table}[!htbp]
  \begin{tabular}[c]{|r|r|r|r|r|r|}
    \hline
    symbol & raw rank & alpha & symbol & raw rank  & ret.1.d \\
      \hline
     IBM   &  1  & 1.57  &  HPQ &  1 & -0.063 \\
     MS    &  2  & 1.26  & SUNW &  2 & -0.056 \\
     EBAY  &  3  & 1.24  &  AET &  3 & -0.041 \\
     CBBO  &  4  & 1.21  & YHOO &  4 & -0.036 \\
     SCHW  &  5  & 1.15  &    T &  5 & -0.014 \\
     PAYX  &  6  & 1.12  &  CVX &  6 & -0.011 \\
     HAL   &  7  & 1.12  & GOOG &  7 & -0.011 \\
     AMD   &  8  & 1.10  & PAYX &  8 & -0.002 \\
     MSFT  &  9  & 0.99  & CBBO &  9 &  0.003 \\
     CVX   & 10  & 0.96  &  HAL & 10 &  0.009 \\
     AET   & 11  & 0.92  & QCOM & 11 &  0.011 \\
     HPQ   & 12  & 0.81  & EBAY & 12 &  0.014 \\
     QCOM  & 13  & 0.77  & SCHW & 13 &  0.029 \\
     GOOG  & 14  & 0.65  & AAPL & 14 &  0.036 \\
     YHOO  & 15  & 0.64  &   MS & 15 &  0.041 \\
     \hline  
  \end{tabular}
  \caption[alpha and one-day return ranks]{The alpha and one-day
  returns of candidates suggest different rank orderings.  All of the
  candidates are buys.\label{theory table 1}}
\end{table}
      
Table \ref{theory table 1} has a row for each of 15 candidates, their
alpha and one-day return values, and the raw ranks we would generate
from these values.  All of the candidates are buys so greater alpha
values are better and lesser one-day return values are better.

One portfolio manager might decide that she wants to make trades based
only on alpha.  She chooses the top ten trades according to alpha.  A
second portfolio manager may want to make trades based only on one-day
return.  She chooses the top ten trades according to one-day return.
The third portfolio manager considers both alpha and one-day return
and choose her favorite trades by examining both.  

Portfolio manager three believes in buying equities which have had
price decreases of greater than 4\% during the previous trading day.
Consequently, she would buy HPQ, SUNW, and AET.  She would fill her
remaining orders using the top 7 trades according to alpha.

How would the third portfolio manager write a function that expresses
her trading preferences?  What if some days she acted like the first
portfolio manager and on other days like the second portfolio manager?
How would she account for a change in preference for one of the sorts?

Our solution allows any of these portfolio managers to express her
trading preferences without having to write a function that relates
the different measures of desirability.  Instead, she would use the
weighting function that the \texttt{portfolio} package provides.  She
would examine the trade list created using different weighting schemes
and adjust the weights until the  utility derived from the
last candidate traded was greater than the  cost of the first
trade \emph{not} made.

For example, the portfolio manager may decide that YHOO is a better
reversal trade than the last alpha trade and revise the weighting
scheme so that she makes one less alpha trade and one more reversal
trade.

\begin{table}[!htbp]
  \begin{tabular}[c]{|r|r|r|r|r|r|}
    \hline
    symbol & raw rank & alpha & symbol & raw rank  & ret.1.d \\
      \hline
     IBM   &  1  & 1.57  &  HPQ &  1 & -0.063 \\
     MS    &  2  & 1.26  & SUNW &  2 & -0.056 \\
     EBAY  &  3  & 1.24  &  AET &  3 & -0.041 \\ 
     CBBO  &  4  & 1.21  & YHOO &  4 & -0.036 \\ \cline{5-5}
     SCHW  &  5  & 1.15  &    T &  5 & -0.014 \\ 
     PAYX  &  6  & 1.12  &  CVX &  6 & -0.011 \\ \cline{2-2}
     HAL   &  7  & 1.12  & GOOG &  7 & -0.011 \\ 
     AMD   &  8  & 1.10  & PAYX &  8 & -0.002 \\
     MSFT  &  9  & 0.99  & CBBO &  9 &  0.003 \\
     CVX   & 10  & 0.96  &  HAL & 10 &  0.009 \\
     AET   & 11  & 0.92  & QCOM & 11 &  0.011 \\
     HPQ   & 12  & 0.81  & EBAY & 12 &  0.014 \\
     QCOM  & 13  & 0.77  & SCHW & 13 &  0.029 \\
     GOOG  & 14  & 0.65  & AAPL & 14 &  0.036 \\
     YHOO  & 15  & 0.64  &   MS & 15 &  0.041 \\
     \hline  
  \end{tabular}
  \caption[Trading Preferences II]{Portfolio manager 3 revises her
  trading preferences.\label{theory_table_2}}
\end{table}

What ultimately matters is the last candidate we decide to trade and
the first candidate we decide not to trade.  By using rank orders
instead of underlying values, we do not have to combine the different
sorts.  Instead, we can express our preferences for different,
possibly unrelated criteria through the use of a weighting scheme we
provide in \texttt{portfolio}.  

\subsection{Pairing trades}

Let us return to discussing trade list construction.  In practise,
most equity portfolios must be maintained at a specific market value.
One logical way to achieve this result would be to pair desirable buys
and sells of equal market value, and this is what we do in the
\texttt{portfolio} package.  We call these pairings of buys and sells
a swap:

\begin{description}

  \item{\bf{swap}}: A pairing of a buy and sell or short and cover of
  similar market market value and desirability.

\end{description}

We have already created the framework to create this swaps; we break
the candidates into chunks of similar market value and then rank these
chunks individually.  If our candidate trades included buys and sells,
we would simply match the most desirable buys with the most desirable
sells.  However, our candidate trades are all buys, and we want to
increase the market value of our portfolio by \$1,000.

\subsubsection{Dummy chunks}

If we want to increase the market value of the portfolio, we must buy
more than we sell.  Therefore, we do not want to pair a buy with a
sell.  We just want buys.  The situation where we just want buys or
sells is a special case.  The \texttt{portfolio} package is structured
so that we must also trade in pairs.  To work within the package
framework we introduce the concept of \emph{dummy chunks}:

\begin{description}

  \item{\bf{dummy chunk}}: A \emph{fake} buy or sell chunk that we pair with
  a real buy or sell chunk in situations where we want to increase or
  decrease the market value of the portfolio.

\end{description}

As our example only contains buys, we have paired every buy with a
dummy sell.\protect\footnote{We only show the head of the swaps table.}

\begin{Schunk}
\begin{Soutput}
            tca.rank.enter tca.rank.exit rank.gain
MSFT.1,NA.0            1.9         10000     -9998
MSFT.2,NA.0            1.9         10000     -9998
MSFT.3,NA.0            1.9         10000     -9998
MSFT.4,NA.0            1.9         10000     -9998
GM.1,NA.0              1.6         10000     -9998
SCHW.1,NA.0            1.3         10000     -9999
\end{Soutput}
\end{Schunk}

In the table above, the row names express the chunk ticker symbols
that form the swap.  To the left of the comma is an enter chunk, and
to the right of the comma is an exit chunk.\protect\footnote{Enter
chunks are either a buy or short.  A buy allows us to take a long
position and a short allows us to take a short position.  Exit chunks
are either sells or covers.  A sell allows us to exit a long position
and a cover allows us to exit a short position.}  The exit chunks all
have a symbol \texttt{NA.0} because they are dummy sells.  The
\texttt{tca.rank.enter} column expresses the trade-cost adjusted rank
of the enter chunk, the buy, and the \texttt{tca.rank.exit} column
expresses the trade-cost adjusted rank of the exit chunk, the dummy
sell.  The \texttt{rank.gain} column expresses the difference in
trade-cost adjusted rank between the enter and the exit, the buy and
dummy sell.  

We have spent considerable time discussing the generation of all types
of ranks for buys, but we have not yet discussed ranking sells.  For
sells, better ranks are more negative.  Therefore, a great sell might
have a synthetic rank of -3.5.  In section \ref{}, we discuss how
we generate the ranks for the sells.  For now, just note that better
sells have more negative ranks.

Recall that our goal is to make the trades which yield the most
utility.  In spending our \$1,000, we want to trade the best chunks.
So that we make the best buys when increasing the market value of the
portfolio, we assign the dummy sells an arbitrarily high rank.  In the
table above, the dummy sells have a trade-cost adjusted rank of
-10,000.  We match the best the buys and sells by calculating rank
gain.  As no real sells will yield the same rank gain that the pairing
of buy and a dummy sell yields, we create pairs with all the dummy
sells before even considering other sells.  As there are no sells in
this example, all the swaps consist of a buy and a dummy sell.

Let's quickly review why we create swaps.  We want to maximise utility
by making the candidate trades or portions of candidate trades that
yield the greatest  utility.  Generally, we want to maintain
the portfolio equity at a constant level.  A logical way to do this
involves pairing buys and sells of similar market value.  To maximise
utility, we should pair the most best ranked buys and sells.  In
special cases, we want to increase or decrease the market value
of our portfolio.  In order to do this, we must make more of one type
of trade.  However, this would require that we have swaps that contain
only a buy or sell.  Since we cannot have a swap of only one trade, we
introduce dummy trades.  As dummy trades have an arbitrarily high
synthetic rank they pair with the best buys and sells to ensure that
we choose the most useful candidates in changing the market value of
the portfolio.

\subsection{Accounting for turnover}

\emph{Please note that this and subsequent sections need to account
  for change in turnover application.  Now all swaps are done such
  that the total market value of trades goes up to but doesn't exceed
  the turnover amount.  In the meantime I have adjusted the example's
  turnover to \$2,000 so that at least one chunk is done, although now
  Sweave chunks will be inconsistent with the text.}

As we stated earlier, holding period would be endogenous if
we could always set it to maximise risk-adjusted return.  However,
most real world portfolios have a set holding period and consequently,
a set turnover.  There is no real concept of turnover or holding
period in this example.  We have \$1,000 to invest in our portfolio
over the course of a single day.  Although this additional investment
does not represent turnover, we can view our \$1,000 as representing a
daily turnover of \$1,000.  We want to make the best ranked trades
until the cumulative market value of these trades exceeds the money we
have to invest.  Analogously, we would say that we want to make the
best ranked trades until we exceed turnover.

As our turnover in this example is \$2000, all of our
trades will not have a market value greater than
\$2000:

\begin{Schunk}
\begin{Soutput}
            tca.rank.enter tca.rank.exit rank.gain
MSFT.1,NA.0            1.9         10000     -9998
\end{Soutput}
\end{Schunk}

MSFT is the the best ranked trade.  Consequently, we choose swaps of
MSFT before choosing other swaps.  We make
1 because each swap has a value of
approximately \$2000, and our turnover is
\$2000.

\subsection{Actual orders}

We do not want to submit two orders for 8 shares of MSFT.  Before
submitting the trade list, we must roll-up the swaps into larger
orders.  We first remove the dummy chunks:

\begin{Schunk}
\begin{Soutput}
       side   mv alpha ret.1.d rank.t chunk.shares chunk.mv tca.rank
MSFT.1    B 6000   1.5   -0.02    1.9           33     1980      1.9
\end{Soutput}
\end{Schunk}

Then we combine the chunks to form a single order per candidate:

\begin{Schunk}
\begin{Soutput}
     side shares   mv alpha ret.1.d rank.t
MSFT    B     33 1980   1.5   -0.02    1.9
\end{Soutput}
\end{Schunk}

We now have an order for 33 shares of
MSFT, which is the sum of the chunks of
MSFT.  Having discussed in words the process of
trade list creation, we describe, step-by-step, the process of
building a \texttt{tradelist} object in R.


%% \section{A simple example}

%% \SweaveOpts{echo=FALSE, quiet=TRUE}

%% Assume that we already have a small portfolio consisting of positions
%% in various equities.  We have been given an additional \$1,000 to
%% invest in the current portfolio, and we must invest this \$1,000 over
%% the course of one trading day.  We only consider buys.  This is not
%% a realistic scenario, but it is simple.

%% Recall our first simplifying assumption that we already have a
%% ``target portfolio'', an ideal set of holdings that we would
%% immediately switch to if trading were free.  We use the \$1,000 to move
%% towards the target portfolio.

%% %\subsection{Holdings}
%% %\subsubsection{Current portfolio}
%% %\subsubsection{Target portfolio}
%% %\subsubsection{Portfolio difference and candidate trades}

%% \subsection{Current and target holdings}

%% Our current portfolio consists of shares of
%% 3 companies, IBM (International Business
%% Machines), GM (General Motors) and EBAY (EBay).

%% <<p.current@shares, echo=FALSE>>=
%% p.current@shares[, c("shares", "price")]
%% @

%% The \texttt{shares} column expresses how many shares of each stock are in
%% the portfolio, and the \texttt{price} column expresses the most recent
%% price of that equity.\footnote{For simplicity, we express monetary
%% values in this document as US dollars.}  The market value of the
%% current portfolio can be calculated by summing the products of the
%% shares and prices, and it is
%% \$13,000.

%% As per our simplifying assumption the user already has a target
%% portfolio.

%% <<p.target@shares>>=
%% p.target@shares[, c("shares", "price")]
%% @

%% We would like to buy more shares of GM and take positions in SCHW
%% (Charles Schwab Inc.), MSFT (Microsoft), and GOOG (Google).  The
%% market value of the target portfolio is
%% \$28,000.
 
%% The target portfolio expresses the positions we want to take with our
%% \$1,000.  It is an ideal set of holdings that we would immediately
%% switch to if trading were free.  The positions in the target portfolio
%% are all desirable, otherwise we would not switch to them.  However, it
%% is not given that we can move to the target portfolio.  Liquidity,
%% price impact, and turnover complicate trading.  At best, we will move
%% in the direction of the target portfolio.  There are a variety of ways
%% to do this.  The \texttt{portfolio} package helps us to decide which
%% is the best.

%% \subsection{Portolio difference and candidate trades}

%% The portfolio difference may be understood as the trades that would
%% change our current holdings into our ideal holdings. From the
%% portfolio difference, we determine our \emph{candidate trades}.

%% \begin{description}

%% \item{\bf{candidate trades}}: The set of trades to move from
%% the current portfolio to the target portfolio.  If trading were free,
%% we would make all of these trades.

%% \end{description}

%% \subsection{Expressing preferences amongst trades}

%% Part of our simplifying assumption is that we would instantly switch
%% to the target portfolio if trading were free.  This implies that all
%% of the candidate trades are desirable.  However, they are not all
%% equally desirable.  Some trades are better than others.  We want to
%% determine which candidate trades or subsets of the candidate trades
%% yield the most utility.  We might do this by assigning each stock some
%% measure of utility.



%% This approach works fine when we only have one measure of utility, but
%% fails when we have multiple measures of utility.  Let's say that we
%% have two measures of utility, \emph{alpha} and \emph{one-day return}.

%% Alpha is a measure of utility.  Using some quantitative model, we
%% generate a measure of alpha for the stocks in our portfolio.  



%% Each stock has some value associated with it which represents alpha.
%% Positive alpha values indicate that we should buy a stock; the greater
%% the magnitude, the better the buy.  EBAY (EBay) is the best buy and
%% IBM (International Business Machines) is worst buy.  The opposite
%% applies for sells.  Negative alpha values indicate that we should sell
%% stock.

%% Based on these alpha values, we decide to buy all of these stocks.  
 
%% When we associate trades with alpha, we say that we sort by alpha or
%% use alpha as a \emph{sort.}

%% \begin{description}

%% \item{\bf{sort}}: a set of values associated with a set of positions.
%% Higher values suggest positive future performance and negative values
%% suggest poor future performance.  Based on a sort we can determine
%% what type of trade is most desirable for each position.  Therefore, we
%% want to buy or cover positions with positive sort value and sell or
%% short positions with negative sort values.

%% \end{description}

%% Like portfolio construction, alpha generation is beyond the scope of
%% this document; we provide alpha values and will be using alpha as a
%% sort.  We associate good buys with greater alpha values.  The
%% candidate at the head of the data frame has the highest alpha value
%% and is therefore the most desirable trade.

%% \subsection{Preliminary ranks}

%% We determine which trades are most desirable by generating an overall
%% measure of desirability for each trade.  The first step in generating
%% this value involves creating a \emph{rank ordering} of the trades for each
%% sort we have created.

%% \begin{description}

%%   \item{\bf{rank ordering}}: a linear ordering of the candidates,
%%   where each candidate is assigned a rank from the set $1, 2,
%%   3\dots{}n$.  Trade 1 provides the greatest utility and trade $n$
%%   provides the least utility.  

%% \end{description}

%% <<trades ranked and ordered by alpha>>=

%% ## removes the "ret.1.d" column for successful row binding later on
%% alpha <- tl@ranks[,!names(tl@ranks) %in% "ret.1.d"]

%% ## appends a column so we know what sort these values come from
%% alpha$sort <- "alpha"

%% @ 

%% We lose \emph{cardinal information} when we use ranks.  Cardinal
%% information is the set of values we use to create the ranks.  In some
%% cases we may want to use more than one measure of desirability.  We
%% may have more than one source of cardinal information.  Imagine that
%% we want to use both alpha and one-day return as sorts.  If we believe
%% in one day reversal, we would assign higher ranks to both orders to
%% sell stocks with positive one-day returns and to orders to buy stocks
%% with negative one-day returns.  However, we associate more desirable
%% buys with greater sort values.  To account for this, we have taken the
%% inverse of all one-day return values.  Therefore, if the return
%% one-day return for GM appears to be $0.10$, it should really be
%% $-0.10$.  The table on the left shows the one-day return values after
%% we have taken their inverse.  The table to the right contains that
%% actual one-day return values.

%% \begin{verbatim}

%%      rank side ret.1.d         rank side ret.1.d
%% GM      1    B    0.10    GM      1    B   -0.10
%% GOOG    2    B    0.01    GOOG    2    B   -0.01
%% MSFT    3    B   -0.01    MSFT    3    B    0.01
%% SCHW    4    B   -0.02    SCHW    4    B    0.02

%% \end{verbatim}

%% <<ret.1.d sort>>=
%% tmp <- tl@ranks[order(tl@ranks$ret.1.d), c("side","ret.1.d")]
%% tmp <- cbind(rank = 1:nrow(tmp), tmp)
%% tmp$ret.1.d <- tmp$ret.1.d[order(tmp$ret.1.d, decreasing = TRUE)]
%% row.names(tmp) <- tl@candidates$id
%% @

%% MSFT has the highest rank according to one-day return
%% because it has the most negative return of all the buys, but for the
%% \texttt{portfolio} package to properly process our one-day reversal
%% sort, we must take the inverse of all the one-day return values.

%% \subsubsection{The problem of multiple sorting criteria}

%% When we combine the sorts in a single data frame, it is not clear
%% which sort values we should use. If we order by alpha we get the
%% following set of ranks:

%% <<two sorts sorting by alpha>>=
%% tmp.1 <- tl@ranks[order(tl@ranks$alpha, decreasing = TRUE), c("alpha", "ret.1.d")]
%% tmp.1 <- tmp.1 <- cbind(rank = 1:nrow(tmp.1), tmp.1)
%% tmp.1
%% @ 

%% Ordering by one-day return yields another ordering:

%% <<two sorts sorting by ret.1.d>>=
%% tmp.2 <- tl@ranks[order(tl@ranks$ret.1.d, decreasing = TRUE), c("alpha", "ret.1.d")]
%% tmp.2 <- cbind(rank = 1:nrow(tmp.2), tmp.2)
%% tmp.2
%% @

%% When we use multiple sorts, there is no obvious way by hich we would
%% order the trades by desirability.  When sorting by alpha,
%% MSFT is the most desirable trade, and when
%% sorting by one-day return, GM is the most
%% desirable trade.  We cannot easily compare or combine the two sorts
%% because we do not know what the exact relationship is between one-day
%% reversal and alpha.  Neither sort is even on the same numeric scale.
%% Should we alternate between using values from alpha and one-day
%% return?  How would we decide how often to alternate between the sorts?
%% The way in which we express preferences amongst trades in the
%% \texttt{portfolio} package represents our answer to these questions.
%% In sections \ref{words weighting sorts} through \ref{generating
%% synthetic ranks} we discuss our method for ranking trades when we
%% multiple measures of desirability.  In section \ref{sorting theory} we
%% will discuss the reasoning behind our mtehod of ordering trades.

%% <<trades ranked and ordered by ret.1.d>>=
%% ## we don't actually show any of these values right here
%% ## for buys, ranks by the inverse because lower values are better
%% tl@ranks$rank <- rank(-abs(tl@ranks$ret.1.d), na.last = TRUE, ties.method = "random")
%% ## removes the "alpha" column for successful row binding later on
%% ret.1.d <- tl@ranks[,!names(tl@ranks) %in% "alpha"]
%% ## appends a column so we know what sort these values come from
%% ret.1.d$sort <- "ret.1.d"
%% @

%% %% As per our second simplifying assumption, we lose a certain amount of
%% %% data when we replace cardinal information with ranks.  In doing this
%% %% the \texttt{portfolio} package makes the important assumption, that
%% %% all our different criteria for trading can be captured with a rank
%% %% ordering.  Anyone who uses the package should be aware of this
%% %% assumption.  This assumption and the assumption that the user provides
%% %% a target portfolio are the most significant assumptions we make.

%% %% However, if we did not only consider ranks, it would be difficult to
%% %% compare sorts.  We would have to force the user to provide a function
%% %% that expresses the relationship between the sorts.  Writing such a
%% %% function is hard, especially if the we use multiple sorts.  To
%% %% facilitate the comparison of sorts, we introduce a weighting scheme.

%% \subsection{Weighting sorts}
%% \label{words weighting sorts}

%% As sorts express preferences amongst stocks, weights express
%% preferences amongst sorts.  A weight is a measure of how important a
%% sort is relative to other sorts.  Say that alpha and one-day return
%% are equally important to us.  We assert this by assigning them both
%% the same weight.  Assigning a weight divides the sort values by that
%% weight.  We assign a weight of one.

%% <<weighted ranks>>=
%% ## saves off alpha$rank
%% alpha.rank.orig <- alpha$rank

%% ## creates a column w/ 1-day return
%% alpha$ret.1.d <- ret.1.d$ret.1.d

%% ## orders each column individually

%% alpha[order(alpha$rank), c("alpha","ret.1.d")]
%% @ 

%% However, the ranks for one-day return remain the same because one-day
%% return has a weight of 1.

%% <<unweighted ret.1.d>>=

%% @ 

%% Having divided the original, \emph{raw ranks} by weight, we now have
%% \emph{weighted ranks}.

%% \begin{description}

%% \item{\bf{raw ranks}}: the original, linearly spaced ranks, built on
%%   the scale $1, 2, 3\dots{}n$

%% \item{\bf{weighted ranks}}: the raw ranks divided by the weights of
%%   the sorts.

%% \end{description}

%% We now have two ranks associated with each candidate, one from the
%% alpha sort and another from the one-day return sort.  To illustrate
%% that we have duplicate ranks for each sort, we bind the
%% equally-weighted alpha and one-day return sorts to form a single data
%% frame.

%% <<unweighted ranks>>=

%% ## sets the ranks of alpha to the original, unweighted ranks
%% alpha$rank <- alpha.rank.orig

%% ## subsets out the "alpha" and "ret.1.d" columns so that both data frames have the same set of columns
%% alpha   <- alpha[,!names(alpha) %in% "alpha"]
%% ret.1.d <- ret.1.d[,!names(ret.1.d) %in% "ret.1.d"]

%% overall.ranks <- rbind(alpha, ret.1.d)
%% overall.ranks <- overall.ranks[order(overall.ranks$rank), c("id", "sort", "rank", "side", "shares", "mv")]
%% row.names(overall.ranks) <- paste(overall.ranks$id, overall.ranks$sort, sep = ".")
%% overall.ranks[, c("rank", "sort", "side", "shares", "mv")]
%% @ 

%% Equally weighted sorts produce candidate trades from each sort at every
%% rank.  Using $x$ sorts produces $x$ ranks for each candidate.  We
%% assign each candidate the best rank generated for it by any sort:


%% <<unweighted w/out duplicates>>=

%% ranks <- alpha
%% top.ranks <- aggregate(overall.ranks[c("rank")], by = list(id = overall.ranks$id), min)
%% ranks$rank <- top.ranks$rank[match(ranks$id, top.ranks$id)]
%% ranks[order(ranks$rank), c("rank", "shares", "mv")]
%% @ 

%% Both GM and MSFT have been assigned a rank of one.  This occurs
%% because MSFT has been ranked 1 by the alpha sort and GM has been
%% ranked 1 by the one-day return sort.  SCHW has been ranked 2 by the
%% alpha sort and GOOG has been ranked 3 by the alpha sort.

%% When we equally weight the sorts we are equally likely to use ranks
%% from either sort.  This behavior is logical because assigning sorts
%% equal weights suggests that they are equally important.  However, the
%% sorts may not always be equally important.  In the next example we use
%% a weighting scheme that causes us to use one sort to the exlusion of
%% the other.

%% To ignore one-day return we make alpha 10 times more important.  We
%% consider 10 ranks from alpha for every one rank from one-day return.
%% As there are only 4 candidate trades, we
%% choose the rankings in alpha over the ranks in the one-day return
%% sort.

%% <<unbalanced weights>>=
%% ## Assigns one sort, alpha, a much higher weight than the other sort

%% ## restores the original alpha rankings

%% alpha$rank <- alpha.rank.orig

%% ## weights the alpha rankings by 10

%% alpha$rank <- alpha$rank / 10

%% overall.ranks <- data.frame()
%% overall.ranks <- rbind(alpha, ret.1.d)
%% overall.ranks <- overall.ranks[order(overall.ranks$rank), c("id", "sort", "rank", "side", "shares", "mv")]
%% row.names(overall.ranks) <- paste(overall.ranks$id, overall.ranks$sort, sep = ".")
%% overall.ranks[c("rank", "side", "shares", "mv")]
%% @

%% This extreme weighting stacks the alpha ranks on top of the one-day
%% return ranks.  Since we always assign the lowest rank from all trades
%% to a sort, we consider the alpha ranks before any of the one-day
%% return ranks.

%% <<unbalanced w/out duplicates>>=

%% ranks <- alpha
%% top.ranks <- aggregate(overall.ranks[c("rank")], by = list(id = overall.ranks$id), min)
%% ranks$rank <- top.ranks$rank[match(ranks$id, top.ranks$id)]
%% ranks[order(ranks$rank), c("rank", "sort", "shares", "mv")]
%% @

%% Making the alpha 10 times as important as the one-day return sort
%% causes us to only use ranks from the alpha sort.  We do not even
%% consider the number 1 ranked one-day return trade until we examine all
%% the alpha sorts ranked in the top ten.

%% The last weighting we consider falls somewhere in between the
%% previous two.

%% <<mixed weighting>>=

%% ## returns alpha$rank to original level

%% alpha$rank <- alpha.rank.orig
%% alpha$rank <- alpha$rank / 1.5

%% overall.ranks <- data.frame()
%% overall.ranks <- rbind(alpha, ret.1.d)
%% overall.ranks <- overall.ranks[order(overall.ranks$rank), c("id", "sort", "rank", "side", "shares", "mv")]
%% row.names(overall.ranks) <- paste(overall.ranks$id, overall.ranks$sort, sep = ".")
%% overall.ranks[c("rank", "side", "shares", "mv")]
%% @

%% Having assigning a weight of 1.5 to alpha we divide each alpha value
%% by 1.5.

%% <<mixed w/out duplicates>>=

%% ranks <- alpha
%% top.ranks <- aggregate(overall.ranks[c("rank")], by = list(id = overall.ranks$id), min)
%% ranks$rank <- top.ranks$rank[match(ranks$id, top.ranks$id)]
%% tmp <- ranks[order(ranks$rank), c("rank", "shares", "mv")]
%% tmp
%% @

%% We use three of the ranks from the alpha sort and one
%% rank from the one-day return sort.  This is the weighting scheme that
%% we use in the rest of the example.

%% To review, the ranking process has four steps.  First, we ranked each
%% trade according to both alpha and one-day return to generate raw
%% ranks.  Second, we weight these ranks.  Third, we combined the alpha
%% and one-day return ranks.  Fourth, we eliminated duplicates by
%% associating each trade with the lowest rank assigned to it by either
%% alpha or one-day return.  We call these ranks preliminary ranks
%% because they are not the final values we use to determine the
%% desirability of each trade.  Nonetheless, we must generate preliminary
%% ranks before we can arrive at final ranks, the calculation of which we
%% describe in the next section.

%% \subsection{Generating synthetic ranks}
%% \label{generating synthetic ranks}

%% If trade 1 is $X$ better than the trade 2, then is trade 99 $X$ better
%% than trade 100?  Most portfolio managers would argue that the
%% difference in utility between trade 1 and trade 2 is greater than the
%% difference in utility between trade 99 and trade 100.  However, with
%% raw ranks, we make no assertion of how much better one trade is than
%% another trade.  To express the tendency for us to derive more utility
%% from the most highly ranked trades, we synthesise yet another set of
%% values from the weighted ranks.  We call these values synthetic
%% ranks.\protect\footnote{We are abusing the term ``ranks'' by using it
%% in several different contexts.}

%% \begin{description}

%% \item{\bf{synthetic ranks}}: values generated by mapping the weighted
%% ranks to a truncated normal distribution ($> 85^{th}$ percentile on
%% $N(0,1)$).

%% \end{description}

%% First, we re-rank the weighted ranks:

%% <<>>=
%% tmp$rank <- rank(tmp$rank, ties.method = "first")
%% tmp[order(tmp$rank),]
%% @ 

%% Next, we evenly distribute the ranks on the interval interval
%% $[0.85,1)$ such that the best ranked trades are closest to 1 and the
%% worst ranked trades are closest to $0.85$:

%% <<scaling ranks>>=

%% ## a hacked version of the scaling function in calcRanks, built only
%% ## for a list of all buys

%% r.max  <- max(tmp$rank) + 1
%% r.mult <- 0.15
%% r.add  <- 0.85

%% tmp$rank.s <- (r.mult * tmp$rank[nrow(tmp):1] / r.max) + r.add

%% ## Saves off rank.s for later use
%% rank.s <- tmp

%% tmp
%% @ 

%% We list the scaled ranks in \texttt{rank.s}.  Next, we map to the a
%% truncated normal distribution.\protect\footnote{$> 85^{th}$ percentile
%% of $N(0,1)$}


%% <<synthesised ranks>>=
%% tmp$rank.t <- qnorm(tmp$rank.s)
%% tmp[,c("rank", "rank.s", "rank.t", "shares", "mv")]
%% @

%% The \texttt{rank.t} column lists the ranks mapped to the truncated
%% normal distribution.  GM has the best rank and
%% SCHW, has the lowest rank.  We might
%% expect to see a \texttt{rank.t} of approximately 3.5 for the best
%% ranked trade, but because we only have 4 candidates
%% and the scaled values are evenly spaced on the interval $[0.85,1)$,
%% the normalised value of the best ranked trade is not as great as
%% it would be if we had 100 trades.

%% \emph{preliminary ranks section}
%% We mentioned in section \ref{} that synthetic ranks express the
%% greater differences in utility between adjacent, highly ranked trades
%% ($1,2,3\dots{}$) than between adjacent, poorly ranked trades: \\*

%% \begin{table}[!htbp]
%%   \begin{tabular}[c]{|rr|rr|rr|}
%%     \hline
%%     rank & $\Delta$ & $N(0,1)$ & $\Delta$ & $> 85^{th}$ of $(0,1)$ & $\Delta$ \\
%%       \hline
%%       1   & 1  &  3.50 & 1.17 & 3.50 & 0.53 \\
%%       2   & 1  &  2.32 & 0.27 & 2.96 & 0.21 \\
%%       3   & 1  &  2.05 & 0.17 & 2.74 & 0.13 \\
%%       4   & 1  &  1.88 & 0.13 & 2.61 & 0.10 \\
%%       5   & 1  &  1.75 & -    & 2.51 & -    \\
%%       .   & .  &     . &    . &    . &    . \\
%%       .   & .  &     . &    . &    . &    . \\
%%       48  & 1  &  0.05 & 0.03 & 1.46 & 0.01 \\
%%       49  & 1  &  0.02 & 0.02 & 1.45 & 0.01 \\
%%       50  & 1  &  0.00 & 0.02 & 1.44 & 0.01 \\
%%       51  & 1  & -0.02 & 0.02 & 1.43 & 0.01 \\
%%       52  & 1  & -0.05 & -    & 1.41 & -    \\
%%       .   & .  &     . &    . &    . &    . \\
%%       .   & .  &     . &    . &    . &    . \\
%%       96  & 1  & -1.64 & 0.11 & 1.06 & 0.00 \\
%%       97  & 1  & -1.75 & 0.13 & 1.06 & 0.00 \\
%%       98  & 1  & -1.88 & 0.17 & 1.06 & 0.00 \\
%%       99  & 1  & -2.05 & 0.27 & 1.06 & 0.00 \\
%%       100 & -  & -2.32 & -    & -    & NA   \\
%%       \hline
%%   \end{tabular}
%%   \caption[Synthetic rank distributions]{Creating synthetic ranks
%%     using a linear distribution, a normal distribution, and a
%%     truncated normal distribution.  Delta columns express the
%%     difference in utility between adjacent trades.\label{distribution
%%     table}}
%% \end{table}

%% Table \ref{distribution table} expresses the differences amongst
%% distributions we might use to rank 100 trades.  The \texttt{Raw Rank}
%% column contains the raw ranks for the 5 best trades, the 5
%% middle-ranked trades, and the 5 worst trades.  In this example the
%% ranks on $[1,100]$ are spaced at intervals of one.  The alpha
%% difference between every trade is the same.  If we use raw rank as a
%% measure of alpha, we derive the same utility from every trade.  Trade
%% 1 is one better than trade 2, and trade 99 is one better than trade
%% 100.

%% The normal distribution column $(N(0,1))$ expresses what happens when
%% we normalise the raw ranks.  The normal distribution correctly
%% expresses our belief that there is a large  difference in
%% alpha between the best ranked trades.  However, use of the normal
%% distribution would incorrectly suggest that there are similarly large
%% alpha differences between the worst trades.  We get these results when
%% using the normal distribution because the best and worst ranked trades
%% form the tails of the distribution.  We do not want large 
%% differences in alpha amongst the worst rank trades.  The 
%% alpha differences decrease until we reach trade 50, then increase
%% again as we move towards the other tail of the distribution.  We want
%% alpha to remain the same on the margin past the 50th trade.

%% To address the problems of a normal and linear distribution, we use a
%% truncated normal distribution, $> 85^{th} \% of N(0,1)$.  In the
%% right\-most delta $(\Delta)$ column, the  alpha differences
%% between the best ranked trades is over 50 times greater than the
%%  alpha differences between the worst ranked trades.  Every
%% trade ranked worse than 50 has a similar  alpha difference.
%% Although the subset $[0.85,1)$ is slightly arbitrary, (we could have
%% set the lower extreme to be 0.84, 0.85, or another similar value) it
%% serves our purpose of expressing large differences in  alpha
%% and where we find the best buys, at one tail, and small differences in
%%  alpha amongst the worst buys.

%% Recall the steps we have taken towards in generating our final measure
%% of rank, synthetic rank.  First, we converted the sort values to raw
%% ranks.  Second, we converted the raw ranks to weighted ranks.  Third,
%% we scaled the weighted ranks to $[0.85,1)$ to generate scaled weights.
%% Lastly, we mapped the scaled weights to the truncated normal
%% distribution.  By only using the $85^{th}$ percentile and above, we
%% express our belief that the differences in alpha between the best
%% ranked trades is much greater than the differences in alpha between
%% the worst ranked trades.

%% If the costs associated with trading any stock, all things being
%% equal, were the same, we would not care about the difference in
%% utility between any trades.  We would move down the trade list from
%% best to worst until we match the allotted turnover.  However, our
%% trading influences prices and may reduce the desirability of a trade.


%% \subsection{Chunks, synthetic rank, and trade-cost adjustment}
%% \label{Chunks, synthetic rank, and trade-cost adjustment}

%% We want to know at what point the cost of trading an equity exceeds
%% the utility of trading that equity.  In the \texttt{portfolio}
%% package, we use synthetic rank to represent utility.  Determining the
%% cost of purchasing an additional share is impossible if our smallest
%% trading unit is an entire order so we break each order into
%% \emph{chunks}.

%% \begin{description}

%% \item{\bf{chunk}}: A portion of a candidate trade.

%% \end{description}

%% We break candidate trades into chunks by market value.  Each chunk has
%% a market value of approximately \$2000:

%% <<chunk w/out tca.rank>>=
%% tl@chunks[, c("side", "shares", "mv", "alpha", "ret.1.d", "rank.t", "chunk.shares", "chunk.mv")]
%% @ 

%% The candidate trades are broken into 10 chunks.
%% The number following the period in the row name expresses the chunk
%% number for that particular equity.  The \texttt{chunks.mv} column
%% expresses the market value of each chunk. The \texttt{chunk.shares}
%% column expresses how many shares each chunk consists of.

%% \subsubsection{Trade-cost adjustment of individual chunks}

%% As we trade greater percentages of the average daily volume, the price
%% of the equity will increase.  To reflect this phenomenon, we penalise
%% the synthetic ranks of the chunk as we trade greater percentages of
%% the daily volume.  We call this penalty \emph{trade-cost adjustment}.

%% \begin{description}

%% \item{\bf{trade-cost adjustment}}: Lowering a chunk's rank because of
%% trading volume.

%% \end{description}

%% To fix this idea, let's first examine the daily volumes of our
%% candidate trades.\protect\footnote{The \texttt{volume} column
%% represents some measure of past trading volume such as the average
%% trading volume over the last 30 days.  A daily measure of
%% \texttt{volume} is not required; we would use whatever measure is
%% natural for the frequency with which we trade.}

%% <<trading volume>>=
%% trading.volume <- data.frame()
%% trading.volume <- cbind(rank.t = tl@ranks[, c("rank.t")], volume = tl@data[match(tl@ranks$id, tl@data$id), c("volume")], shares = tl@ranks[, "shares"])
%% row.names(trading.volume) <- tl@ranks$id
%% trading.volume
%% @

%% \emph{This must be updated as I change the portfolio}

%% The trades we want to make for both MSFT, SCHW, and GOOG involve less
%% than 3\% of the daily trading volume.  However, we must trade 100\% of
%% the daily trading volume.  We would probably not be able to purchase
%% all these shares in one day, and even if we could, we would affect
%% prices significantly.  Moving into the position over several days
%% would be better.

%% We use a trade-cost adjustment function to express how increasing
%% trade costs reduce the desirability of candidate trades.  To better
%% approximate utility, we penalise synthetic ranks at the chunks level.
%% Doing this allows us to better determine at which point the cost of
%% trading an additional chunk is greater than the utility derived by
%% trading an additional chunk.  We perform trade-cost adjustment on the
%% chunks by keeping track of what percentage of the daily volume we have
%% traded with each additional chunk.  The first chunk to cross the
%% threshold of 15\% of the daily trading volume is penalised by a fixed
%% amount.  All subsequent chunks are penalised by that amount, and any
%% further chunks that pass 30\% or 45\% percent of the daily trading
%% volume receive further penalties.  The chunks of GM, an illiquid
%% equity, have been trade-cost adjusted.

%% <<chunks w/ tca.rank>>=

%% head(tl@chunks[, c("side", "mv", "alpha", "ret.1.d",
%%               "rank.t", "chunk.shares", "chunk.mv", "tca.rank")])
%% @ 

%% The \texttt{tca.rank} column expresses the synthetic rank adjusted for
%% trade-cost.  As the only candidate for which we want to purchase more
%% than 15\% of the daily trading volume is GM, it is the only candidate
%% for which we trade-cost adjust the chunks.  Every chunk of GM beyond
%% the first has been trade-cost adjusted.  This will cause us to
%% consider the chunks of other candidate trades before we trade
%% additional chunks of GM:

%% <<ordered chunks w/ tca.rank>>=

%% tl@chunks[order(tl@chunks$tca.rank, decreasing = TRUE), c("side",
%%               "mv", "alpha", "ret.1.d", "rank.t", "chunk.shares",
%%               "chunk.mv", "tca.rank")]
%% @ 

%% As MSFT is the best ranked candidate and does not receive a trade-cost
%% penalty, we would trade all the shares of MSFT before considering the
%% other candidates.\protect\footnote{Assuming that derived turnover is
%% greater than the market value of all the candidate trades.}  Having
%% completed all the trades of MSFT, we would consider the first chunk of
%% GM, the only chunk which has not been trade-cost adjusted.
%% Subsequently, we would trade all the chunks of SCHW and GOOG, the
%% candidate trades ranked 3 and 4.  Lastly, we trade the penalised
%% chunks of GM.

%% \subsubsection{Synthetic rank and trade-cost adjustment of small portfolios}
%% \label{Synthetic rank and trade-cost adjustment of small portfolios}
%% In this example, trade-cost adjustment decreases the desirability of
%% the chunks of GM in a non-trivial way.  Although GM is ranked 2nd as a
%% candidate trade, every other candidate trade would be made before we
%% completed all the chunks of GM.  When we consider such a small number
%% of trades, we assume that all of the trades are of approximately equal
%% quality; the  difference in utility between candidate trades
%% is fairly small.  This occurs because the scaled ranks are evenly
%% distributed on $[0.85,1)$:

%% <<rank.s>>=
%% rank.s
%% @ 

%%   When we only have 4 candidates, none of
%% the scaled ranks will be very close to $1$, and consequently, none of
%% the synthetic ranks will fall at the extreme tail
%% of the normal distribution:

%% <<rank.t>>=
%% rank.t <- rank.s
%% rank.t$rank.t <- qnorm(rank.t$rank.s)
%% rank.t
%% @ 

%% Consequently, the  difference in utility between candidate
%% trades will be small when there are few candidate trades.
%% Heuristically, this seems correct because if we are making very few
%% trades, we would most likely derive similar utility from any of
%% them.\protect\footnote{This does not exclude our expressing a
%% preference amongst the sorts.}  Therefore, it makes sense for us to
%% trade the other three candidates if the costs associated with trading
%% GM are non\-trivial.

%% \subsubsection{Synthetic rank and trade-cost adjustment of large portfolios}
%% \label{Synthetic rank and trade-cost adjustment of large portfolios}

%% Moving away from our example for a moment, imagine that we have a
%% large current and target portfolio, the trade list for which contains
%% 100 candidate trades.  When we have a large portfolio, we tend to
%% view the differences in  utility between candidates in the
%% manner we described in section \ref{}.  When evenly distribute the
%% scaled ranks on the interval $[0.85,1)$, we have more ranks at
%% the extreme tail:

%% <<misc$rank.s>>=
%% misc$rank.s
%% @ 

%% The row names express the equity ticker symbols.  \texttt{rank} is the
%% raw rank. \texttt{rank.s} is the scaled rank, and \texttt{rank.t} is
%% the synthetic rank.  The best ranked trade
%% IBM, has a scaled rank value very
%% close to one, , and
%% a synthetic rank close to three.  This indicates that the best rank
%% falls at the tail of the normal distribution.  The worst ranked
%% candidates not only have low synthetic ranks, but they also have very
%% small  differences in synthetic rank.  If we trade-cost adjust
%% one of the poorly ranked candidates we will most likely not trade it
%% until we have traded all other candidates not penalised by trade cost
%% adjustment.  On the other hand, we would still trade
%% IBM,
%% GOOG, or
%% GM, even if some of the chunks had
%% been trade-cost adjusted:

%% Here we have a subset of the hypothetical chunk table for the 100
%% candidate example.  For this example, the GOOG candidate has been
%% broken up into 2 chunks and the IBM candidate has been broken up into
%% 4 chunks.  The ranks of 2$^{nd}$, 3$^{rd}$, and
%% 4$^{th}$ chunks of IBM have been penalised for trade costs.
%% Therefore, we trade the first chunk of IBM, followed by all the chunks
%% of GOOG.  Subsequently, we trade the remaining chunks of IBM because
%% the trade-cost adjusted rank of its chunks is still greater than the
%% un-penalised synthetic rank of the next most desirable candidate, GM.

%% Let's quickly review how we generate the final, synthetic ranks.  The
%% preliminary values from which we draw the raw ranks are the sorts we
%% define.  In this example, we defined sorts for alpha and one-day
%% return.  In creating raw ranks abstract away the underlying values
%% provided by the sorts.  At this point, we still have a different set
%% of raw ranks for each sorts.  To express preferences amongst the
%% sorts, we apply weights to the sorts.  This step yields weighted
%% ranks.  From the sets of weighted ranks, we associate with each
%% candidate the best weighted rank from any sort.  Next, we scale the
%% buys to the interval $[0.85,1)$.  This step yields scaled ranks.  From
%% scaled ranks, we generate synthetic ranks by mapping the scaled ranks
%% to a truncated normal distribution.  Next, we break the candidates
%% into chunks and perform trade-cost adjustment as necessary.  This
%% yields trade-cost adjusted ranks which are the final measure of a
%% chunks desirability.

%% \subsection{Sorting theory}
%% \label{sorting theory}

%% Chooing the best candidate when we have multiple measures of
%% desirability is difficult.  Consider the situation where we must
%% choose ten stocks to trade.

%% Assuming that we use some type of formula to generate alpha, we might
%% be able to incorporate our other sorts into the formula for alpha.
%% Instead of having alpha and one-day return as distinct sorts, we would
%% only have one sort, alpha, which would also take one-day return into
%% account.  For this to work, however, we would have to write a function
%% that accounted for the the ordering of every trade by every sort.
%% Furthermore, this function would have to take into account our
%% preference for certain certain sorts over other sorts.  To elaborate
%% on the difficulty of this creating such a function, let us consider
%% the situation where we must choose our ten favourite trades, in no
%% particular order, using the data in the table below.

%% \begin{table}[!htbp]
%%   \begin{tabular}[c]{|r|r|r|r|r|r|}
%%     \hline
%%     symbol & raw rank & alpha & symbol & raw rank  & ret.1.d \\
%%       \hline
%%      IBM   &  1  & 1.57  &  HPQ &  1 & -0.063 \\
%%      MS    &  2  & 1.26  & SUNW &  2 & -0.056 \\
%%      EBAY  &  3  & 1.24  &  AET &  3 & -0.041 \\
%%      CBBO  &  4  & 1.21  & YHOO &  4 & -0.036 \\
%%      SCHW  &  5  & 1.15  &    T &  5 & -0.014 \\
%%      PAYX  &  6  & 1.12  &  CVX &  6 & -0.011 \\
%%      HAL   &  7  & 1.12  & GOOG &  7 & -0.011 \\
%%      AMD   &  8  & 1.10  & PAYX &  8 & -0.002 \\
%%      MSFT  &  9  & 0.99  & CBBO &  9 &  0.003 \\
%%      CVX   & 10  & 0.96  &  HAL & 10 &  0.009 \\
%%      AET   & 11  & 0.92  & QCOM & 11 &  0.011 \\
%%      HPQ   & 12  & 0.81  & EBAY & 12 &  0.014 \\
%%      QCOM  & 13  & 0.77  & SCHW & 13 &  0.029 \\
%%      GOOG  & 14  & 0.65  & AAPL & 14 &  0.036 \\
%%      YHOO  & 15  & 0.64  &   MS & 15 &  0.041 \\
%%      \hline  
%%   \end{tabular}
%%   \caption[alpha and one-day return ranks]{The alpha and one-day
%%   returns of candidates suggest different rank orderings.  All of the
%%   candidates are buys.\label{theory table 1}}
%% \end{table}
      
%% Table \ref{theory table 1} has a row for each of 15 candidates, their
%% alpha and one-day return values, and the raw ranks we would generate
%% from these values.  All of the candidates are buys so greater alpha
%% values are better and lesser one-day return values are better.

%% One portfolio manager might decide that she wants to make trades based
%% only on alpha.  She chooses the top ten trades according to alpha.  A
%% second portfolio manager may want to make trades based only on one-day
%% return.  She chooses the top ten trades according to one-day return.
%% The third portfolio manager considers both alpha and one-day return
%% and choose her favorite trades by examining both.  

%% Portfolio manager three believes in buying equities which have had
%% price decreases of greater than 4\% during the previous trading day.
%% Consequently, she would buy HPQ, SUNW, and AET.  She would fill her
%% remaining orders using the top 7 trades according to alpha.

%% How would the third portfolio manager write a function that expresses
%% her trading preferences?  What if some days she acted like the first
%% portfolio manager and on other days like the second portfolio manager?
%% How would she account for a change in preference for one of the sorts?

%% Our solution allows any of these portfolio managers to express her
%% trading preferences without having to write a function that relates
%% the different measures of desirability.  Instead, she would use the
%% weighting function that the \texttt{portfolio} package provides.  She
%% would examine the trade list created using different weighting schemes
%% and adjust the weights until the  utility derived from the
%% last candidate traded was greater than the  cost of the first
%% trade \emph{not} made.

%% For example, the portfolio manager may decide that YHOO is a better
%% reversal trade than the last alpha trade and revise the weighting
%% scheme so that she makes one less alpha trade and one more reversal
%% trade.

%% \begin{table}[!htbp]
%%   \begin{tabular}[c]{|r|r|r|r|r|r|}
%%     \hline
%%     symbol & raw rank & alpha & symbol & raw rank  & ret.1.d \\
%%       \hline
%%      IBM   &  1  & 1.57  &  HPQ &  1 & -0.063 \\
%%      MS    &  2  & 1.26  & SUNW &  2 & -0.056 \\
%%      EBAY  &  3  & 1.24  &  AET &  3 & -0.041 \\ 
%%      CBBO  &  4  & 1.21  & YHOO &  4 & -0.036 \\ \cline{5-5}
%%      SCHW  &  5  & 1.15  &    T &  5 & -0.014 \\ 
%%      PAYX  &  6  & 1.12  &  CVX &  6 & -0.011 \\ \cline{2-2}
%%      HAL   &  7  & 1.12  & GOOG &  7 & -0.011 \\ 
%%      AMD   &  8  & 1.10  & PAYX &  8 & -0.002 \\
%%      MSFT  &  9  & 0.99  & CBBO &  9 &  0.003 \\
%%      CVX   & 10  & 0.96  &  HAL & 10 &  0.009 \\
%%      AET   & 11  & 0.92  & QCOM & 11 &  0.011 \\
%%      HPQ   & 12  & 0.81  & EBAY & 12 &  0.014 \\
%%      QCOM  & 13  & 0.77  & SCHW & 13 &  0.029 \\
%%      GOOG  & 14  & 0.65  & AAPL & 14 &  0.036 \\
%%      YHOO  & 15  & 0.64  &   MS & 15 &  0.041 \\
%%      \hline  
%%   \end{tabular}
%%   \caption[Trading Preferences II]{Portfolio manager 3 revises her
%%   trading preferences.\label{theory_table_2}}
%% \end{table}

%% What ultimately matters is the last candidate we decide to trade and
%% the first candidate we decide not to trade.  By using rank orders
%% instead of underlying values, we do not have to combine the different
%% sorts.  Instead, we can express our preferences for different,
%% possibly unrelated criteria through the use of a weighting scheme we
%% provide in \texttt{portfolio}.  

%% \subsection{Pairing trades}

%% Let us return to discussing trade list construction.  In practise,
%% most equity portfolios must be maintained at a specific market value.
%% One logical way to achieve this result would be to pair desirable buys
%% and sells of equal market value, and this is what we do in the
%% \texttt{portfolio} package.  We call these pairings of buys and sells
%% a swap:

%% \begin{description}

%%   \item{\bf{swap}}: A pairing of a buy and sell or short and cover of
%%   similar market market value and desirability.

%% \end{description}

%% We have already created the framework to create this swaps; we break
%% the candidates into chunks of similar market value and then rank these
%% chunks individually.  If our candidate trades included buys and sells,
%% we would simply match the most desirable buys with the most desirable
%% sells.  However, our candidate trades are all buys, and we want to
%% increase the market value of our portfolio by \$1,000.

%% \subsubsection{Dummy chunks}

%% If we want to increase the market value of the portfolio, we must buy
%% more than we sell.  Therefore, we do not want to pair a buy with a
%% sell.  We just want buys.  The situation where we just want buys or
%% sells is a special case.  The \texttt{portfolio} package is structured
%% so that we must also trade in pairs.  To work within the package
%% framework we introduce the concept of \emph{dummy chunks}:

%% \begin{description}

%%   \item{\bf{dummy chunk}}: A \emph{fake} buy or sell chunk that we pair with
%%   a real buy or sell chunk in situations where we want to increase or
%%   decrease the market value of the portfolio.

%% \end{description}

%% As our example only contains buys, we have paired every buy with a
%% dummy sell.\protect\footnote{We only show the head of the swaps table.}

%% <<swaps table>>=
%% head(tl@swaps[, c("tca.rank.enter", "tca.rank.exit",
%% "rank.gain")])
%% @ 

%% In the table above, the row names express the chunk ticker symbols
%% that form the swap.  To the left of the comma is an enter chunk, and
%% to the right of the comma is an exit chunk.\protect\footnote{Enter
%% chunks are either a buy or short.  A buy allows us to take a long
%% position and a short allows us to take a short position.  Exit chunks
%% are either sells or covers.  A sell allows us to exit a long position
%% and a cover allows us to exit a short position.}  The exit chunks all
%% have a symbol \texttt{NA.0} because they are dummy sells.  The
%% \texttt{tca.rank.enter} column expresses the trade-cost adjusted rank
%% of the enter chunk, the buy, and the \texttt{tca.rank.exit} column
%% expresses the trade-cost adjusted rank of the exit chunk, the dummy
%% sell.  The \texttt{rank.gain} column expresses the difference in
%% trade-cost adjusted rank between the enter and the exit, the buy and
%% dummy sell.  

%% We have spent considerable time discussing the generation of all types
%% of ranks for buys, but we have not yet discussed ranking sells.  For
%% sells, better ranks are more negative.  Therefore, a great sell might
%% have a synthetic rank of -3.5.  In section \ref{}, we discuss how
%% we generate the ranks for the sells.  For now, just note that better
%% sells have more negative ranks.

%% Recall that our goal is to make the trades which yield the most
%% utility.  In spending our \$1,000, we want to trade the best chunks.
%% So that we make the best buys when increasing the market value of the
%% portfolio, we assign the dummy sells an arbitrarily high rank.  In the
%% table above, the dummy sells have a trade-cost adjusted rank of
%% -10,000.  We match the best the buys and sells by calculating rank
%% gain.  As no real sells will yield the same rank gain that the pairing
%% of buy and a dummy sell yields, we create pairs with all the dummy
%% sells before even considering other sells.  As there are no sells in
%% this example, all the swaps consist of a buy and a dummy sell.

%% Let's quickly review why we create swaps.  We want to maximise utility
%% by making the candidate trades or portions of candidate trades that
%% yield the greatest  utility.  Generally, we want to maintain
%% the portfolio equity at a constant level.  A logical way to do this
%% involves pairing buys and sells of similar market value.  To maximise
%% utility, we should pair the most best ranked buys and sells.  In
%% special cases, we want to increase or decrease the market value
%% of our portfolio.  In order to do this, we must make more of one type
%% of trade.  However, this would require that we have swaps that contain
%% only a buy or sell.  Since we cannot have a swap of only one trade, we
%% introduce dummy trades.  As dummy trades have an arbitrarily high
%% synthetic rank they pair with the best buys and sells to ensure that
%% we choose the most useful candidates in changing the market value of
%% the portfolio.

%% \subsection{Accounting for turnover}

%% As we stated in section \ref{}, holding period would be endogenous if
%% we could always set it to maximise risk-adjusted return.  However,
%% most real world portfolios have a set holding period and consequently,
%% a set turnover.  There is no real concept of turnover or holding
%% period in this example.  We have \$1,000 to invest in our portfolio
%% over the course of a single day.  Although this additional investment
%% does not represent turnover, we can view our \$1,000 as representing a
%% daily turnover of \$1,000.  We want to make the best ranked trades
%% until the cumulative market value of these trades exceeds the money we
%% have to invest.  Analogously, we would say that we want to make the
%% best ranked trades until we exceed turnover.

%% As our turnover in this example is \$2000, all of our
%% trades will not have a market value greater than
%% \$2000:

%% <<>>=
%% tl@swaps.actual[, c("tca.rank.enter", "tca.rank.exit",
%% "rank.gain")]
%% @ 

%% MSFT is the the best ranked trade.  Consequently, we choose swaps of
%% MSFT before choosing other swaps.  We make
%% 1 because each swap has a value of
%% approximately \$2000, and our turnover is
%% \$2000.

%% \subsection{Actual orders}

%% We do not want to submit two orders for 8 shares of MSFT.  Before
%% submitting the trade list, we must roll-up the swaps into larger
%% orders.  We first remove the dummy chunks:

%% <<remove idiots>>=

%% tl@chunks.actual[, c("side", "mv", "alpha", "ret.1.d", "rank.t",
%%               "chunk.shares", "chunk.mv", "tca.rank")]
%% @ 

%% Then we combine the chunks to form a single order per candidate:

%% <<>>=
%% tl@actual[, !names(tl@actual) %in% c("id")]
%% @

%% We now have an order for 33 shares of
%% MSFT, which is the sum of the chunks of
%% MSFT.  Having discussed in words the process of
%% trade list creation, we describe, step-by-step, the process of
%% building a \texttt{tradelist} object in R.

\section{Creating a long-only tradelist in R}
\label{a long-only tradelist}




To create a \texttt{tradelist}, we need four main pieces.  The first
two pieces necessary to create a \texttt{tradelist} are
\texttt{portfolio} objects.  One of these portfolios is our current
portfolio.

Our current portfolio is a superset of the previous holdings.  The
major difference between the two portfolios is that the current
portfolio in this example includes positions that we sell.  This
\texttt{portfolio}, named \texttt{p.current}, consists of
6 positions and has a market value of
\$47,750.


\begin{Schunk}
\begin{Sinput}
> p.current.shares
\end{Sinput}
\begin{Soutput}
     shares price
IBM     100    10
GM      100    30
EBAY     75   120
DELL     50   110
QCOM     75   190
AMD     150   100
\end{Soutput}
\end{Schunk}

The target portfolio is a superset of the previous target portfolio.
It contains 6 positions and has a market
value of \$47,500.


\begin{Schunk}
\begin{Sinput}
> p.target.shares
\end{Sinput}
\begin{Soutput}
     shares price
GOOG     50    20
EBAY     75   120
IBM     100    10
GM      200    30
SCHW    100    50
MSFT    100    60
AMD     100   100
QCOM     50   190
\end{Soutput}
\end{Schunk}

We calculate the portfolio difference to determine the candidate
trades.\protect\footnote{The data frame is a subset of the
\texttt{candidates} data frame.  We often take subsets of data frames
so that they fit better on the page.  If we do so we indicate this by
prepending the name of the data frame with \texttt{sub}.}


\begin{Schunk}
\begin{Sinput}
> sub.candidates
\end{Sinput}
\begin{Soutput}
     orig target side shares    mv
AMD   150    100    S     50 -5000
DELL   50      0    S     50 -5500
GM    100    200    B    100  3000
GOOG    0     50    B     50  1000
MSFT    0    100    B    100  6000
QCOM   75     50    S     25 -4750
SCHW    0    100    B    100  5000
\end{Soutput}
\end{Schunk}

The candidate buys are the same as before and we have 3 candidate
sells.  The market value is signed and expresses the net effect a
candidate has on the dollar value of a portfolio.  

\subsection{Assigning weights}

We assign weights to the sorts by creating a list.

\begin{Schunk}
\begin{Sinput}
> sorts <- list(alpha = 1, ret.1.d = 1.1)
\end{Sinput}
\end{Schunk}

We assign a weight of 1 to alpha and a weight of 1.1 to one-day
return.

\subsection{Passing additional information to \texttt{tradelist}}
\label{paitt}

The fourth item is a data frame.  The \texttt{portfolio} package
requires that this data frame contain columns for \texttt{id},
\texttt{volume}, \texttt{price.usd}, and the sorts:


\begin{Schunk}
\begin{Sinput}
> sub.data
\end{Sinput}
\begin{Soutput}
       id volume price.usd alpha ret.1.d
IBM   IBM   2100        10 -0.76  -0.003
GOOG GOOG   2200        20  1.10   0.010
GM     GM    100        30  1.00   0.100
SCHW SCHW   2500        50  1.20  -0.010
MSFT MSFT   2600        60  1.50  -0.020
AMD   AMD   3000       100 -0.94   0.010
DELL DELL   3100       110 -0.15   0.070
EBAY EBAY   3200       120 -0.32   0.001
QCOM QCOM   3900       190 -0.36  -0.005
\end{Soutput}
\end{Schunk}

\texttt{volume} expresses some measure of average trading volume.
\texttt{price.usd} is the most recent price of the security in US
dollars. We must also include the sorts we define in \texttt{sorts},
\texttt{alpha} and \texttt{ret.1.d}.

\subsection{Calling \texttt{new}}
\label{lo new}

We use \texttt{p.current}, \texttt{p.target}, the \texttt{sorts}, and
\texttt{data} as arguments to \texttt{new}.

\begin{Schunk}
\begin{Sinput}
> tl <- new("tradelist", orig = p.current, target = p.target, 
+     chunk.usd = 2000, sorts = sorts, turnover = 30250, data = data)
\end{Sinput}
\end{Schunk}


In this call, the \texttt{new} method for \texttt{tradelist} accepts 8
parameters:\footnote{The \texttt{new} method of \texttt{tradelist} can
accept more parameters, but they are optional.}  The first argument,
\texttt{"tradelist"}, specifies the name of the object that we want to
create.  The argument to the \texttt{orig} parameter,
\texttt{p.current}, is the current portfolio.  The argument to the
\texttt{target} parameter, \texttt{p.current}, is the target
portfolio.  The \texttt{sorts} parameter accepts the \texttt{sorts}
list we created earlier.  We create chunks with a granularity of of
\$2,000.  The \texttt{data}
parameter accepts the data frame we created earlier with columns for
\texttt{id}, \texttt{volume}, \texttt{price.usd}, and the sorts.

The \texttt{turnover} parameter accepts an integer argument which
expresses the maximum market value all orders made in one session.  In
the previous example we only had \$1,000 with which we could buy
stocks.  In this example, we can both buy and sell equities.  We might
sell an equity and use the proceeds to buy another equity.  However,
the turnover restriction applies to sells just as much as buys.  If we
have a turnover of \$1,000, we may make \$1,000 worth of buys, \$1,000
worth of sells, or something in between.  For this example, we have
set the turnover equal to the unsigned market value of all the
candidate trades.  This means that we take the absolute value of all
market values, which is \$30,250.  Having
set \texttt{turnover} to this value, we complete every candidate
trade.


We have demonstrated how to create a simple \texttt{tradelist} in R.
In the next section we examine the \texttt{tradelist} that we have
constructed.  In doing so, we learn how the \texttt{tradelist}
generation algorithm works.

\section{The \texttt{tradelist} algorithm}
\label{the tradelist algorithm}
                                                                                  
The \texttt{tradelist} code provides an algorithm, divisible into
seven smaller steps, that generates a set of trades that will move the
current, original portfolio towards an ideal, target portfolio.  The
seven steps in the algorithm correspond to the following methods of
the \texttt{tradelist} class: \texttt{calcCandidates},
\texttt{calcRanks}, \texttt{calcChunks}, \texttt{calcSwaps},
\texttt{calcSwapsActual}, \texttt{calcChunksActual}, and
\texttt{calcActual}. 

The user never needs to directly call any of these methods when using
the \texttt{portfolio} package.  A call to the \texttt{new} method of
the \texttt{tradelist} class invokes the \texttt{initialize} method of
\texttt{tradelist}.  The \texttt{initialize} method then calls the
seven methods serially.  The first step of the \texttt{tradelist}
algorithm involves determining which types of orders we must make in
order to trade towards the target portfolio.

\subsection{The \texttt{calcCandidates} method}
\label{the calcCandidates method}

As stated in our simplifying assumption, we only consider trades that
bring us closer to the target portfolio. To determine candidate trades
we calculate which positions have changed.  If a position has changed,
we determine what type of trade the candidate is (buy or sell) by
taking the portfolio difference to generate a list of candidate
trades.

\begin{Schunk}
\begin{Sinput}
> tl@candidates
\end{Sinput}
\begin{Soutput}
       id orig target side shares    mv
AMD   AMD  150    100    S     50 -5000
DELL DELL   50      0    S     50 -5500
GM     GM  100    200    B    100  3000
GOOG GOOG    0     50    B     50  1000
MSFT MSFT    0    100    B    100  6000
QCOM QCOM   75     50    S     25 -4750
SCHW SCHW    0    100    B    100  5000
\end{Soutput}
\end{Schunk}

Given the data stored in the \texttt{candidates} data frame and the
\texttt{data} data frame, the \texttt{portfolio} package can generate
the trade list.

\subsection{The \texttt{calcRanks} Method}
\label{calcranks}
\label{the calcRanks method}

Ranking the trades is possibly the most complicated task delegated to
the \texttt{tradelist} class.  When the rank-generating algorithm
returns, the \texttt{ranks} data frame \texttt{tradelist} will contain
the synthetic rank, \texttt{rank.t}, for each trade.

\subsubsection{Interpretation of sort values}

When we define a sort, we express our preference for purchasing
different stocks.  Lesser values express a preference for selling or
shorting a position and greater values express a preference for buying
or covering a position.  In the previous example we only saw positive
alpha values because all the candidates were buys.  If the values were
not positive, we might question why the trade was even a candidate.
Recall our first simplifying assumption that all of the can\-di\-dates
are de\-sir\-able and the \texttt{portfolio} package only helps us to
determine which are the most desirable.

In real life, we want to create a sort using meaningful values that
express our trading preferences.  One such value is one-day return.

\subsubsection{Creating raw ranks for a long-only portfolio}
\label{Creating raw ranks for a long-only portfolio}

The first step in creating ranks is generating raw ranks.  We break
the trades into separate data frames by side and rank the trades
within each side because one type of trade is no than another type
of trade.  

\begin{Schunk}
\begin{Soutput}
$B
       id orig target side shares   mv ret.1.d rank
GM     GM  100    200    B    100 3000    0.10    1
GOOG GOOG    0     50    B     50 1000    0.01    2
SCHW SCHW    0    100    B    100 5000   -0.01    3
MSFT MSFT    0    100    B    100 6000   -0.02    4

$S
       id orig target side shares    mv ret.1.d rank
QCOM QCOM   75     50    S     25 -4750  -0.005    1
AMD   AMD  150    100    S     50 -5000   0.010    2
DELL DELL   50      0    S     50 -5500   0.070    3
\end{Soutput}
\end{Schunk}

The \texttt{\$B} data frame shows the buys ranked with other buys and
the \texttt{\$S} data frame shows the sells ranked with other sells.
The most desirable buys are those associated with the greatest values
in \texttt{ret.1.d}.  The most desirable sells are those associated
with the least value in \texttt{ret.1.d}.  Therefore,
GM ranked 1 amongst buys, is the most desirable
buy, and QCOM, ranked 1 amongst sells, is the most
desirable sell.\footnote{We have taken the inverse of all the one-day
return values so that the \texttt{portfolio} package interprets them
correctly.  If we believe one-day reversal, the best buys have
negative one-day returns and the best sells have positive one-day
returns.  Buy low, sell high.  However, the \texttt{portfolio} package
interprets greater values as indicative of the best buys and lesser
values as indicate of the best sells.}

\subsubsection{Interleaving}
\label{interleaving}

We now have two tables of ranks and there are still multiple trades at
each rank: a buy and sell ranked number one, number two and so on.
Combining the two tables of ranks by type leaves us with duplicates:

\begin{Schunk}
\begin{Soutput}
     orig target side shares    mv ret.1.d rank
GM    100    200    B    100  3000   0.100    1
QCOM   75     50    S     25 -4750  -0.005    1
GOOG    0     50    B     50  1000   0.010    2
AMD   150    100    S     50 -5000   0.010    2
SCHW    0    100    B    100  5000  -0.010    3
DELL   50      0    S     50 -5500   0.070    3
MSFT    0    100    B    100  6000  -0.020    4
\end{Soutput}
\end{Schunk}

We argue that there is no natural way to choose between the best buy
and best sell.  To deal with this ambiguity, we always break ties in
rank between a buy and sell by assigning the buy the higher rank.  In
the following table, we create new raw ranks to eliminate the
duplicates.

\begin{Schunk}
\begin{Soutput}
     orig target side shares    mv alpha rank
MSFT    0    100    B    100  6000  1.50    1
AMD   150    100    S     50 -5000 -0.94    2
SCHW    0    100    B    100  5000  1.20    3
QCOM   75     50    S     25 -4750 -0.36    4
GOOG    0     50    B     50  1000  1.10    5
DELL   50      0    S     50 -5500 -0.15    6
GM    100    200    B    100  3000  1.00    7
\end{Soutput}
\end{Schunk}

Notice that each candidate has a unique rank and that the rows
alternate between buy and sell candidates.  The best ranked candidate
trade is a buy because we broke the tie for first between the best
ranked buy and sell by assigning the buy the higher rank.  This
pattern repeats throughout the data frame because we have ties at
every rank except the last.  We call this process of alternating
between the best ranked buys and sells \emph{interleaving}.

\begin{description}

\item{\bf{interleaving}}: The process of breaking the trades up by
  side and ranking them with other trades of the same type, thereby
  yielding multiple trades at each rank.  We always break ties in rank
  with the following ordering: Buys, Sells, Covers, Shorts (B, S, C,
  X).

\end{description}

\subsubsection{Weighted ranks}
\label{lo weighted ranks}

Having interleaved the candidates, we divide the new raw ranks by the
weight assigned to one-day return, 1.1.

\begin{Schunk}
\begin{Soutput}
       id orig target side shares    mv ret.1.d rank
GM     GM  100    200    B    100  3000   0.100 0.83
QCOM QCOM   75     50    S     25 -4750  -0.005 1.65
GOOG GOOG    0     50    B     50  1000   0.010 2.48
AMD   AMD  150    100    S     50 -5000   0.010 3.31
SCHW SCHW    0    100    B    100  5000  -0.010 4.13
DELL DELL   50      0    S     50 -5500   0.070 4.96
MSFT MSFT    0    100    B    100  6000  -0.020 5.79
\end{Soutput}
\end{Schunk}

We assigned alpha a weight of 1 so the ranks remain the same.  

\begin{Schunk}
\begin{Sinput}
> tl@rank.sorts[["alpha"]]
\end{Sinput}
\begin{Soutput}
       id orig target side shares    mv alpha rank
MSFT MSFT    0    100    B    100  6000  1.50    1
AMD   AMD  150    100    S     50 -5000 -0.94    2
SCHW SCHW    0    100    B    100  5000  1.20    3
QCOM QCOM   75     50    S     25 -4750 -0.36    4
GOOG GOOG    0     50    B     50  1000  1.10    5
DELL DELL   50      0    S     50 -5500 -0.15    6
GM     GM  100    200    B    100  3000  1.00    7
\end{Soutput}
\end{Schunk}

We combine the alpha and one-day return ranks into a single data
frame.


\begin{Schunk}
\begin{Soutput}
     id orig target side shares    mv rank
1   AMD  150    100    S     50 -5000  2.0
2   AMD  150    100    S     50 -5000  3.6
3  DELL   50      0    S     50 -5500  6.0
4  DELL   50      0    S     50 -5500  5.5
5    GM  100    200    B    100  3000  7.0
6    GM  100    200    B    100  3000  0.9
7  GOOG    0     50    B     50  1000  5.0
8  GOOG    0     50    B     50  1000  2.7
9  MSFT    0    100    B    100  6000  1.0
10 MSFT    0    100    B    100  6000  6.4
11 QCOM   75     50    S     25 -4750  4.0
12 QCOM   75     50    S     25 -4750  1.8
13 SCHW    0    100    B    100  5000  3.0
14 SCHW    0    100    B    100  5000  4.5
\end{Soutput}
\end{Schunk}

To remove duplicates, we assign each candidate the best weighted rank
associated with it by any sort.  


\begin{Schunk}
\begin{Soutput}
     orig target side shares    mv
GM    100    200    B    100  3000
MSFT    0    100    B    100  6000
QCOM   75     50    S     25 -4750
AMD   150    100    S     50 -5000
GOOG    0     50    B     50  1000
SCHW    0    100    B    100  5000
DELL   50      0    S     50 -5500
\end{Soutput}
\end{Schunk}

And we re-rank the candidates.

\begin{Schunk}
\begin{Soutput}
     target side shares    mv rank.t
GM      200    B    100  3000    1.9
MSFT    100    B    100  6000    1.6
QCOM     50    S     25 -4750   -1.8
AMD     100    S     50 -5000   -1.4
GOOG     50    B     50  1000    1.3
SCHW    100    B    100  5000    1.2
DELL      0    S     50 -5500   -1.2
\end{Soutput}
\end{Schunk}

\subsubsection{Mapping to the truncated normal distribution}

Having weighted the ranks we create synthetic ranks from a truncated
normal distribution.  When we only have buys, we scale the weighted
ranks to $[0.85,1)$.  This gives us the positive tail of the normal
distribution.  We associate more negative values with better sells so
we want to map sells to the negative tail of the normal distribution.
To do this, we scale sells to the interval $(0,0.15]$.

\begin{Schunk}
\begin{Soutput}
     side alpha ret.1.d rank rank.ws
QCOM    S -0.36  -0.005  1.8   0.037
AMD     S -0.94   0.010  2.0   0.075
DELL    S -0.15   0.070  5.5   0.112
SCHW    B  1.20  -0.010  3.0   0.880
GOOG    B  1.10   0.010  2.7   0.910
MSFT    B  1.50  -0.020  1.0   0.940
GM      B  1.00   0.100  0.9   0.970
\end{Soutput}
\end{Schunk}

We map the scaled ranks to the normal distribution.


\begin{Schunk}
\begin{Sinput}
> tl.ranks
\end{Sinput}
\begin{Soutput}
       id orig target side shares    mv alpha ret.1.d rank.t
QCOM QCOM   75     50    S     25 -4750 -0.36  -0.005   -1.8
AMD   AMD  150    100    S     50 -5000 -0.94   0.010   -1.4
DELL DELL   50      0    S     50 -5500 -0.15   0.070   -1.2
SCHW SCHW    0    100    B    100  5000  1.20  -0.010    1.2
GOOG GOOG    0     50    B     50  1000  1.10   0.010    1.3
MSFT MSFT    0    100    B    100  6000  1.50  -0.020    1.6
GM     GM  100    200    B    100  3000  1.00   0.100    1.9
\end{Soutput}
\end{Schunk}

\texttt{rank.t} expresses the synthetic rank.  All of the
sells have a negative \texttt{rank.t} because they have been mapped to
the negative tail of the normal distribution, while all of the buys
have a positive \texttt{rank.t} because they have been mapped to the
other tail.  As described in section \ref{Synthetic rank and
trade-cost adjustment of large portfolios}, the synthetic ranks do not
fall at the extreme tail of the normal distribution.  

\subsection{The \texttt{calcChunks} Method}
\label{calcChunks lo}

Having calculated synthetic ranks, the \texttt{portfolio} package
creates the chunks table.  We defined the market value of each chunk
by specifying the \texttt{chunk.usd} parameter in the call to
\texttt{new}.  The addition of sells does not have a dramatic effect
on the manner in which we generate the chunk table besides
contributing negative trade-cost adjusted ranks.


\begin{Schunk}
\begin{Sinput}
> sub.chunks
\end{Sinput}
\begin{Soutput}
       side rank.t chunk.shares chunk.mv tca.rank
AMD.1     S   -1.4           20    -2000     -1.4
AMD.2     S   -1.4           20    -2000     -1.4
AMD.3     S   -1.4           10    -1000     -1.4
DELL.1    S   -1.2           18    -1980     -1.2
DELL.2    S   -1.2           18    -1980     -1.2
DELL.3    S   -1.2           14    -1540     -1.2
GM.1      B    1.9           67     2010      1.9
GM.2      B    1.9           33      990     -4.1
GOOG.1    B    1.3           50     1000      1.3
MSFT.1    B    1.6           33     1980      1.6
MSFT.2    B    1.6           33     1980      1.6
MSFT.3    B    1.6           33     1980      1.6
MSFT.4    B    1.6            1       60      1.6
QCOM.1    S   -1.8           11    -2090     -1.8
QCOM.2    S   -1.8           11    -2090     -1.8
QCOM.3    S   -1.8            3     -570     -1.8
SCHW.1    B    1.2           40     2000      1.2
SCHW.2    B    1.2           40     2000      1.2
SCHW.3    B    1.2           20     1000      1.2
\end{Soutput}
\end{Schunk}

Most chunks have an unsigned market value of approximately
\$2,000.  The only chunks of
market value significantly less than
\$2,000 are the final chunks of
a candidate.  These chunks are the remainders left after dividing the
rest of the order into \$2,000
chunks.



If we order the chunks by \texttt{tca.rank}, the second chunk of GM
has been severely penalised for trade costs.

\begin{Schunk}
\begin{Sinput}
> head(sub.chunks[order(sub.chunks[["tca.rank"]]), ])
\end{Sinput}
\begin{Soutput}
       side rank.t chunk.shares chunk.mv tca.rank
GM.2      B    1.9           33      990     -4.1
QCOM.1    S   -1.8           11    -2090     -1.8
QCOM.2    S   -1.8           11    -2090     -1.8
QCOM.3    S   -1.8            3     -570     -1.8
AMD.1     S   -1.4           20    -2000     -1.4
AMD.2     S   -1.4           20    -2000     -1.4
\end{Soutput}
\end{Schunk}

GM has a more negative \texttt{tca.rank} than any of the buys or sells,
indicating that this is the last chunk we would trade.

\subsection{The \texttt{calcSwaps} Method}
\label{calcSwaps long-only}

The \texttt{calcSwaps} works in as it did in the previous example, the
main difference being that we pair real buy chunks with real sell
chunks.  We determine which trades to pair for a swap by
calculating \emph{rank gain}. 

\begin{description}

\item{\bf{rank gain}}: The difference in \texttt{tca.rank} between a
buy and a sell.  As the most desirable buys have a very positive
\texttt{tca.rank} and the most desirable sells have a very negative
\texttt{tca.rank}, the best swaps have great \texttt{rank.gain}
values.

\end{description}

Buys with high \texttt{tca.rank} have been matched with sells with low
\texttt{tca.rank}.


\begin{Schunk}
\begin{Sinput}
> swaps.sub
\end{Sinput}
\begin{Soutput}
              side.enter tca.rank.enter side.exit tca.rank.exit rank.gain
GM.1,QCOM.1            B            1.9         S          -1.8       3.7
MSFT.1,QCOM.2          B            1.6         S          -1.8       3.3
MSFT.2,QCOM.3          B            1.6         S          -1.8       3.3
MSFT.3,AMD.1           B            1.6         S          -1.4       3.0
MSFT.4,AMD.2           B            1.6         S          -1.4       3.0
GOOG.1,AMD.3           B            1.3         S          -1.4       2.8
SCHW.1,DELL.1          B            1.2         S          -1.2       2.4
SCHW.2,DELL.2          B            1.2         S          -1.2       2.4
SCHW.3,DELL.3          B            1.2         S          -1.2       2.4
GM.2,NA.0              B           -4.1         S       10000.0  -10004.1
\end{Soutput}
\end{Schunk}

We have paired almost all of the buy chunks with real sell chunks.
The only buy we have not paired with a real sell chunk is the second
chunk of GM.  As the target portfolio
(\$47,500) has approximately the same
market value as the current portfolio
(\$47,750), we will not introduce any
dummy chunks to account for over or under-investment.  We pair GM with
a dummy chunk only because we have run out of real sell chunks to
match it with.  As we would rather make swaps which contain a real buy
and sell chunk, we assign the dummy sell chunk a poor
\texttt{tca.rank} which yields a low \texttt{rank.gain} value.
Consequently, we will not consider this trade until we have considered
all of the other trades.

\subsection{The \texttt{calcSwapsActual} Method}
\label{calcSwapsActual}

The remaining steps of the \texttt{tradelist} algorithm clean up the
\texttt{tradelist} for final use.  In the \texttt{calcSwapsActual}
method we remove the most poorly ranked swaps that exceed turnover.
When we created the \texttt{tradelist}, we set \texttt{turnover} to be
\$30,250, the unsigned market
value of all the candidate trades.  A \texttt{turnover} of
\$30,250 will allow us to
complete every trade.


\begin{Schunk}
\begin{Sinput}
> sub.swaps.actual
\end{Sinput}
\begin{Soutput}
              side.enter tca.rank.enter side.exit tca.rank.exit rank.gain
GM.1,QCOM.1            B            1.9         S          -1.8       3.7
MSFT.1,QCOM.2          B            1.6         S          -1.8       3.3
MSFT.2,QCOM.3          B            1.6         S          -1.8       3.3
MSFT.3,AMD.1           B            1.6         S          -1.4       3.0
MSFT.4,AMD.2           B            1.6         S          -1.4       3.0
GOOG.1,AMD.3           B            1.3         S          -1.4       2.8
SCHW.1,DELL.1          B            1.2         S          -1.2       2.4
SCHW.2,DELL.2          B            1.2         S          -1.2       2.4
SCHW.3,DELL.3          B            1.2         S          -1.2       2.4
\end{Soutput}
\end{Schunk}

Right now, turnover does not cause any swaps to be dropped because it
is greater than the unsigned market value of all the candidate trades,
which is \$30,250.
  
We can cause some swaps to be dropped by setting \texttt{turnover} to
a value less than \$30,250.


\begin{Schunk}
\begin{Sinput}
> tl@turnover <- 30250 - tl@chunk.usd
\end{Sinput}
\end{Schunk}


When we set turnover to a value equal to one chunk less
(2000 than the difference in market value between the
original and target portfolios, the \texttt{calcSwapsActual} method
excises the swap with the lowest \texttt{tca.rank}.

\begin{Schunk}
\begin{Sinput}
> sub.swaps.actual
\end{Sinput}
\begin{Soutput}
              side.enter tca.rank.enter side.exit tca.rank.exit rank.gain
GM.1,QCOM.1            B            1.9         S          -1.8       3.7
MSFT.1,QCOM.2          B            1.6         S          -1.8       3.3
MSFT.2,QCOM.3          B            1.6         S          -1.8       3.3
MSFT.3,AMD.1           B            1.6         S          -1.4       3.0
MSFT.4,AMD.2           B            1.6         S          -1.4       3.0
GOOG.1,AMD.3           B            1.3         S          -1.4       2.8
SCHW.1,DELL.1          B            1.2         S          -1.2       2.4
SCHW.2,DELL.2          B            1.2         S          -1.2       2.4
\end{Soutput}
\end{Schunk}


We have removed the third chunk of GM from the list.

\subsection{The \texttt{calcChunksActual} Method}
\label{calcChunksActual}

Our \texttt{tradelist} is almost complete, but first we must change
the swaps back into chunks.  In addition, we do not want to include
any orders for dummy chunks, so we will remove those when we turn the
swaps back into chunks.  


\begin{Schunk}
\begin{Sinput}
> sub.chunks.actual
\end{Sinput}
\begin{Soutput}
       side alpha ret.1.d rank.t tca.rank chunk.shares chunk.mv chunk
GM.1      B  1.00   0.100    1.9      1.9           67     2010     1
MSFT.1    B  1.50  -0.020    1.6      1.6           33     1980     1
MSFT.2    B  1.50  -0.020    1.6      1.6           33     1980     2
MSFT.3    B  1.50  -0.020    1.6      1.6           33     1980     3
MSFT.4    B  1.50  -0.020    1.6      1.6            1       60     4
GOOG.1    B  1.10   0.010    1.3      1.3           50     1000     1
SCHW.1    B  1.20  -0.010    1.2      1.2           40     2000     1
SCHW.2    B  1.20  -0.010    1.2      1.2           40     2000     2
SCHW.3    B  1.20  -0.010    1.2      1.2           20     1000     3
QCOM.1    S -0.36  -0.005   -1.8     -1.8           11    -2090     1
QCOM.2    S -0.36  -0.005   -1.8     -1.8           11    -2090     2
QCOM.3    S -0.36  -0.005   -1.8     -1.8            3     -570     3
AMD.1     S -0.94   0.010   -1.4     -1.4           20    -2000     1
AMD.2     S -0.94   0.010   -1.4     -1.4           20    -2000     2
AMD.3     S -0.94   0.010   -1.4     -1.4           10    -1000     3
DELL.1    S -0.15   0.070   -1.2     -1.2           18    -1980     1
DELL.2    S -0.15   0.070   -1.2     -1.2           18    -1980     2
DELL.3    S -0.15   0.070   -1.2     -1.2           14    -1540     3
       dummy.quality
GM.1            <NA>
MSFT.1          <NA>
MSFT.2          <NA>
MSFT.3          <NA>
MSFT.4          <NA>
GOOG.1          <NA>
SCHW.1          <NA>
SCHW.2          <NA>
SCHW.3          <NA>
QCOM.1          <NA>
QCOM.2          <NA>
QCOM.3          <NA>
AMD.1           <NA>
AMD.2           <NA>
AMD.3           <NA>
DELL.1          <NA>
DELL.2          <NA>
DELL.3          <NA>
\end{Soutput}
\end{Schunk}

All of the dummy chunks have been removed.

\subsection{The Final Step: Actual Orders}

In the last step of \texttt{tradelist} generation, we ``roll-up'' the
actual chunks for each security to form one order per security.


\begin{Schunk}
\begin{Sinput}
> tl.actual
\end{Sinput}
\begin{Soutput}
     side shares    mv alpha ret.1.d rank.t
AMD     S     50 -5000 -0.94   0.010   -1.4
DELL    S     50 -5500 -0.15   0.070   -1.2
GM      B     67  2010  1.00   0.100    1.9
GOOG    B     50  1000  1.10   0.010    1.3
MSFT    B    100  6000  1.50  -0.020    1.6
QCOM    S     25 -4750 -0.36  -0.005   -1.8
SCHW    B    100  5000  1.20  -0.010    1.2
\end{Soutput}
\end{Schunk}

No rows for chunks remain in the \texttt{actual} data frame.

\section{A Long-Short Example}


For the most part, the \texttt{portfolio} package treats one-sided and
long-short portfolios similarly.  The major difference is that we now
have to take four types of trades into consideration, buys, sells,
shorts, and covers.

\subsection{Current and target portfolios}

Our current portfolio is a superset of the holdings in the previous
example.  This example's current portfolio includes positions that we
will short and cover.  The current portfolio, \texttt{p.current},
consists of 11 positions and has a market
value of \$16,780.


\begin{Schunk}
\begin{Sinput}
> p.current.shares
\end{Sinput}
\begin{Soutput}
     shares price
IBM     100    10
GM      100    30
AMD     150   100
DELL     50   110
EBAY     75   120
QCOM     75   190
HPQ     -50    15
HAL     -75    20
PAYX   -100    25
TXN     -25    25
YHOO    -10    20
\end{Soutput}
\end{Schunk}

The target portfolio is a superset of the target portfolio we used
in the two previous examples.  It contains all the positions in the
previous target portfolio plus positions that we short or cover.


\begin{Schunk}
\begin{Sinput}
> p.target.shares
\end{Sinput}
\begin{Soutput}
     shares price
IBM     100    10
GOOG     50    20
GM      200    30
SCHW    100    50
MSFT    100    60
AMD     100   100
EBAY     75   120
QCOM     50   190
HPQ    -100    15
HAL     200    20
PAYX    -50    25
APPL    -75    30
TXN     -50    25
YHOO     25    20
\end{Soutput}
\end{Schunk}

The target portfolio, \texttt{p.target}, contains
14 positions and has a market value of
\$44,900.  We assume that we have the
additional funds necessary to increase the market value of the
portfolio.

\subsection{Candidate trades}

We calculate the portfolio difference to determine what the candidate
trades will be:


\begin{Schunk}
\begin{Sinput}
> sub.candidates
\end{Sinput}
\begin{Soutput}
     orig target side shares    mv
AMD   150    100    S     50 -5000
APPL    0    -75    X     75 -2250
DELL   50      0    S     50 -5500
GM    100    200    B    100  3000
GOOG    0     50    B     50  1000
HAL   -75      0    C     75  1500
HPQ   -50   -100    X     50  -750
MSFT    0    100    B    100  6000
PAYX -100    -50    C     50  1250
QCOM   75     50    S     25 -4750
SCHW    0    100    B    100  5000
TXN   -25    -50    X     25  -625
YHOO  -10      0    C     10   200
\end{Soutput}
\end{Schunk}

We now have buy, sell, cover, and short candidates (B, S, C, X).
Buys and covers have positive market values because they increase the
value of the portfolio, and sells and shorts have negative market
values because they decrease the value of the portfolio.  Notice that
all the candidate trades necessary to reach the target positions for
HAL and YHOO are not on the candidate list.  We do not include all the
candidate trades to reach these positions because they involve side
changes.

\subsubsection{Side changes and restrictions}

A side change occurs when a position changes from long to short or
short to long.  The \texttt{portfolio} package does not allow a side
change to occur during a single trading session.\footnote{Writing code
so that we make a side change without creating a box position is hard.
We will address this in future versions of the \texttt{portfolio}
package} For a side change to occur, we must make two types of trades.
We must either sell first, then short, or cover first, then buy.  We
only allow the first of one of these trades to occur during a single
trading session.  The second trade is added to the restricted list so
that it may be performed during a later session.  The two trades that
involve side changes have been added to the \texttt{restricted} list.


\begin{Schunk}
\begin{Sinput}
> tl@restricted
\end{Sinput}
\begin{Soutput}
    id orig target side shares   mv            reason
1  HAL    0    200    B    200 4000 Side change enter
2 YHOO    0     25    B     25  500 Side change enter
\end{Soutput}
\end{Schunk}

We have added the buy candidates for HAL and YHOO to the restricted
data frame so that we do not accidentally enter a box position.  The
\texttt{reason} column explains why these candidates have been added
to \texttt{restricted}.  During this trading session we will attempt
to exit the short positions for HAL and YHOO by covering these
positions.  In a subsequent trading session we will attempt to enter a
long position by buying these equities.

\subsection{Creating sorts and assigning them weights}

Like in the previous example, we name the sorts and assign them
weights by creating a list.

\begin{Schunk}
\begin{Sinput}
> sorts <- list(alpha = 1, ret.1.d = 1/2)
\end{Sinput}
\end{Schunk}

We assigned a weight of 1 to alpha and a weight
of 0.5 to one-day return.

\subsection{Passing additional information to \texttt{tradelist}}

We must pass a data frame with columns for \texttt{id},
\texttt{price.usd}, \texttt{volume}, \texttt{alpha}, and
\texttt{ret.1.d} in the call to \texttt{new}:


\begin{Schunk}
\begin{Sinput}
> sub.data
\end{Sinput}
\begin{Soutput}
       id volume price.usd alpha ret.1.d
IBM   IBM   2100        10 -0.76  -0.003
GOOG GOOG   2200        20  1.10   0.010
GM     GM    100        30  1.00   0.100
SCHW SCHW   2500        50  1.20  -0.010
MSFT MSFT   2600        60  1.50  -0.020
AMD   AMD   3000       100 -0.94  -0.040
DELL DELL   3100       110 -0.15  -0.020
EBAY EBAY   3200       120 -0.32  -0.070
QCOM QCOM   3900       190 -0.36  -0.005
HPQ   HPQ   4000        15 -1.30  -0.002
HAL   HAL   4000        20  1.70   0.001
PAYX PAYX   4000        25  0.53  -0.001
APPL APPL   4000        30 -0.30  -0.090
TXN   TXN   4000        25 -0.50  -0.010
YHOO YHOO   4000        20  1.20  -0.002
\end{Soutput}
\end{Schunk}

Aside from having information about additional equities, this data
frame does not differ greatly from the one we passed to new in section
\ref{lo new}.

\subsection{Calling \texttt{new}}
\label{ls new}

Having gathered the components necessary to build a tradelist
\texttt{tradelist}, we make a call to \texttt{new}:

\begin{Schunk}
\begin{Sinput}
> tl <- new("tradelist", orig = p.current, target = p.target, 
+     chunk.usd = 2000, sorts = sorts, turnover = 36825, data = data)
\end{Sinput}
\end{Schunk}


We pass 8 arguments as parameters to the \texttt{new} method.  The
parameters are similar to those in section \ref{lo new} with the
exception of turnover which we have set to
\$36,825.  The value of the
candidate trades in this example is greater than the value of the
candidate trades in the previous example so we must set
\texttt{turnover} higher if we want to complete all of the candidate
trades.

\section{The \texttt{tradelist} algorithm, long-short}

The way the \texttt{portfolio} package builds a long-short
\texttt{tradelist} is similar to the way it builds a long-only
\texttt{tradelist}.  We will walk through the process
of creating a long-short \texttt{tradelist} with \texttt{portfolio}
and discuss the differences between creating long-only and long-short
trade list.

\subsection{Calculating ranks}
 
We calculate the ranks for a long-short portfolio in much the same way
we do so for a long-only portfolio.  The main difference we must take
into is the need to rank four types of trades with other trades of the
same type.  In previous examples we ranked buys and sells separately.
Now we rank buys, sells, covers, and shorts separately.

\subsubsection{Raw ranks with a long-short \texttt{tradelist}}

As per our third simplifying assumption, we do not favour one type of
trade over another type of trade.  As a consequence, we split and rank
the trades separately.

\begin{Schunk}
\begin{Soutput}
$B
       id orig target side shares   mv alpha rank
MSFT MSFT    0    100    B    100 6000   1.5    1
SCHW SCHW    0    100    B    100 5000   1.2    2
GOOG GOOG    0     50    B     50 1000   1.1    3
GM     GM  100    200    B    100 3000   1.0    4

$C
       id orig target side shares   mv alpha rank
HAL   HAL  -75      0    C     75 1500  1.70    3
YHOO YHOO  -10      0    C     10  200  1.20    7
PAYX PAYX -100    -50    C     50 1250  0.53   11

$S
       id orig target side shares    mv alpha rank
AMD   AMD  150    100    S     50 -5000 -0.94    1
QCOM QCOM   75     50    S     25 -4750 -0.36    2
DELL DELL   50      0    S     50 -5500 -0.15    3

$X
       id orig target side shares    mv alpha rank
HPQ   HPQ  -50   -100    X     50  -750  -1.3    1
TXN   TXN  -25    -50    X     25  -625  -0.5    2
APPL APPL    0    -75    X     75 -2250  -0.3    3
\end{Soutput}
\end{Schunk}

Like on page \pageref{Creating raw ranks for a long-only portfolio},
the \texttt{\$B} data frame shows the buys ranked with other buys and
the \texttt{\$S} data frame shows the sells ranked with other sells.
The \texttt{\$C} and \texttt{\$X} data frames show covers and shorts
ranked with other shorts.

\subsubsection{Interleaving}

The last step left us with 4 sets of ranks, one
for each type of trade.  Up to four trades will share each rank when
we combine these data frames to form a list of overall rankings and
the trades will be interleaved using groups of up to
four.\protect\footnote{Some of the groups may not include one trade of
every type.}

\begin{Schunk}
\begin{Soutput}
       orig target side shares    mv alpha rank
B.MSFT    0    100    B    100  6000  1.50    1
S.AMD   150    100    S     50 -5000 -0.94    1
X.HPQ   -50   -100    X     50  -750 -1.30    1
B.SCHW    0    100    B    100  5000  1.20    2
S.QCOM   75     50    S     25 -4750 -0.36    2
X.TXN   -25    -50    X     25  -625 -0.50    2
B.GOOG    0     50    B     50  1000  1.10    3
C.HAL   -75      0    C     75  1500  1.70    3
S.DELL   50      0    S     50 -5500 -0.15    3
X.APPL    0    -75    X     75 -2250 -0.30    3
B.GM    100    200    B    100  3000  1.00    4
C.YHOO  -10      0    C     10   200  1.20    7
C.PAYX -100    -50    C     50  1250  0.53   11
\end{Soutput}
\end{Schunk}

As per the third simplifying assumption, there is no natural way to
choose between the best buy, sell, cover, or short.  To deal with this
ambiguity, we always break ties in rank between a buy, sell, cover,
and short by assigning the buy the highest rank, the sell the second
highest rank, the cover the third highest rank, and the short the
worst rank:

\begin{Schunk}
\begin{Soutput}
     orig target side shares    mv alpha rank
MSFT    0    100    B    100  6000  1.50    1
AMD   150    100    S     50 -5000 -0.94    2
HAL   -75      0    C     75  1500  1.70    3
HPQ   -50   -100    X     50  -750 -1.30    4
SCHW    0    100    B    100  5000  1.20    5
QCOM   75     50    S     25 -4750 -0.36    6
YHOO  -10      0    C     10   200  1.20    7
TXN   -25    -50    X     25  -625 -0.50    8
GOOG    0     50    B     50  1000  1.10    9
DELL   50      0    S     50 -5500 -0.15   10
PAYX -100    -50    C     50  1250  0.53   11
APPL    0    -75    X     75 -2250 -0.30   12
GM    100    200    B    100  3000  1.00   13
\end{Soutput}
\end{Schunk}

Once again, each candidate has a unique rank and the rows appear in
groups of buys, sells, covers, and shorts.  The pattern repeats
throughout he data frame because we have ties at every rank except for
the last.  There is no tie at the last rank because we have an odd
number of candidates.

\subsubsection{Weighted ranks}

Having interleaved the separate rankings by type, we calculate
weighted ranks.

\begin{Schunk}
\begin{Soutput}
       id orig target side shares    mv alpha rank
MSFT MSFT    0    100    B    100  6000  1.50    1
AMD   AMD  150    100    S     50 -5000 -0.94    2
HAL   HAL  -75      0    C     75  1500  1.70    3
HPQ   HPQ  -50   -100    X     50  -750 -1.30    4
SCHW SCHW    0    100    B    100  5000  1.20    5
QCOM QCOM   75     50    S     25 -4750 -0.36    6
YHOO YHOO  -10      0    C     10   200  1.20    7
TXN   TXN  -25    -50    X     25  -625 -0.50    8
GOOG GOOG    0     50    B     50  1000  1.10    9
DELL DELL   50      0    S     50 -5500 -0.15   10
PAYX PAYX -100    -50    C     50  1250  0.53   11
APPL APPL    0    -75    X     75 -2250 -0.30   12
GM     GM  100    200    B    100  3000  1.00   13
\end{Soutput}
\end{Schunk}

We double the one-day return ranks to reflect that one-day return is
less important than alpha.  (Recall that lesser ranks are better.)

\begin{Schunk}
\begin{Sinput}
> tl@rank.sorts[["ret.1.d"]]
\end{Sinput}
\begin{Soutput}
       id orig target side shares    mv ret.1.d rank
GM     GM  100    200    B    100  3000   0.100    2
AMD   AMD  150    100    S     50 -5000  -0.040    4
HAL   HAL  -75      0    C     75  1500   0.001    6
APPL APPL    0    -75    X     75 -2250  -0.090    8
GOOG GOOG    0     50    B     50  1000   0.010   10
DELL DELL   50      0    S     50 -5500  -0.020   12
PAYX PAYX -100    -50    C     50  1250  -0.001   14
TXN   TXN  -25    -50    X     25  -625  -0.010   16
SCHW SCHW    0    100    B    100  5000  -0.010   18
QCOM QCOM   75     50    S     25 -4750  -0.005   20
YHOO YHOO  -10      0    C     10   200  -0.002   22
HPQ   HPQ  -50   -100    X     50  -750  -0.002   24
MSFT MSFT    0    100    B    100  6000  -0.020   26
\end{Soutput}
\end{Schunk}

We assign each candidate the best weighted rank from either sort.  We
combine the data frame of the candidates ranked by alpha with the data
frame of the candidates ranked by one-day return:


\begin{Schunk}
\begin{Soutput}
     id orig target side shares    mv rank
1   AMD  150    100    S     50 -5000    2
2   AMD  150    100    S     50 -5000    4
3  APPL    0    -75    X     75 -2250   12
4  APPL    0    -75    X     75 -2250    8
5  DELL   50      0    S     50 -5500   10
6  DELL   50      0    S     50 -5500   12
7    GM  100    200    B    100  3000   13
8    GM  100    200    B    100  3000    2
9  GOOG    0     50    B     50  1000    9
10 GOOG    0     50    B     50  1000   10
11  HAL  -75      0    C     75  1500    3
12  HAL  -75      0    C     75  1500    6
13  HPQ  -50   -100    X     50  -750    4
14  HPQ  -50   -100    X     50  -750   24
15 MSFT    0    100    B    100  6000    1
16 MSFT    0    100    B    100  6000   26
17 PAYX -100    -50    C     50  1250   11
18 PAYX -100    -50    C     50  1250   14
19 QCOM   75     50    S     25 -4750    6
20 QCOM   75     50    S     25 -4750   20
21 SCHW    0    100    B    100  5000    5
22 SCHW    0    100    B    100  5000   18
23  TXN  -25    -50    X     25  -625    8
24  TXN  -25    -50    X     25  -625   16
25 YHOO  -10      0    C     10   200    7
26 YHOO  -10      0    C     10   200   22
\end{Soutput}
\end{Schunk}

To remove duplicates, we assign each candidate the best weighted rank
associated with it by any sort.  


\begin{Schunk}
\begin{Soutput}
     orig target side shares    mv
MSFT    0    100    B    100  6000
AMD   150    100    S     50 -5000
GM    100    200    B    100  3000
HAL   -75      0    C     75  1500
HPQ   -50   -100    X     50  -750
SCHW    0    100    B    100  5000
QCOM   75     50    S     25 -4750
YHOO  -10      0    C     10   200
APPL    0    -75    X     75 -2250
TXN   -25    -50    X     25  -625
GOOG    0     50    B     50  1000
DELL   50      0    S     50 -5500
PAYX -100    -50    C     50  1250
\end{Soutput}
\end{Schunk}

Once again we generate raw ranks:

\begin{Schunk}
\begin{Soutput}
     orig target side shares    mv rank
MSFT    0    100    B    100  6000  1.0
AMD   150    100    S     50 -5000  2.5
GM    100    200    B    100  3000  2.5
HAL   -75      0    C     75  1500  4.0
HPQ   -50   -100    X     50  -750  5.0
SCHW    0    100    B    100  5000  6.0
QCOM   75     50    S     25 -4750  7.0
YHOO  -10      0    C     10   200  8.0
APPL    0    -75    X     75 -2250  9.5
TXN   -25    -50    X     25  -625  9.5
GOOG    0     50    B     50  1000 11.0
DELL   50      0    S     50 -5500 12.0
PAYX -100    -50    C     50  1250 13.0
\end{Soutput}
\end{Schunk}

Having created weighted ranks, we prepare for the creation of
synthetic ranks.

\subsubsection{Mapping to the truncated normal distribution}

We create synthetic ranks from by mapping the ranks to a truncated
normal distribution.  We scale buys and covers to the the $85^{th}$
percentile and above and sells and shorts to the $15^{th}$ percentile
and below ($(0, 0.15]\cup[0.85,1)$).

\begin{Schunk}
\begin{Soutput}
     side alpha ret.1.d rank rank.ws
AMD     S -0.94  -0.040    2   0.037
HPQ     X -1.30  -0.002    4   0.043
QCOM    S -0.36  -0.005    6   0.075
APPL    X -0.30  -0.090    8   0.107
TXN     X -0.50  -0.010    8   0.107
DELL    S -0.15  -0.020   10   0.112
GOOG    B  1.10   0.010    9   0.880
PAYX    C  0.53  -0.001   11   0.887
SCHW    B  1.20  -0.010    5   0.910
YHOO    C  1.20  -0.002    7   0.925
GM      B  1.00   0.100    2   0.940
HAL     C  1.70   0.001    3   0.962
MSFT    B  1.50  -0.020    1   0.970
\end{Soutput}
\end{Schunk}

Finally, we map the values to the truncated normal distribution:


\begin{Schunk}
\begin{Sinput}
> tl.ranks
\end{Sinput}
\begin{Soutput}
       id orig target side shares    mv alpha ret.1.d rank.t
AMD   AMD  150    100    S     50 -5000 -0.94  -0.040   -1.8
HPQ   HPQ  -50   -100    X     50  -750 -1.30  -0.002   -1.7
QCOM QCOM   75     50    S     25 -4750 -0.36  -0.005   -1.4
APPL APPL    0    -75    X     75 -2250 -0.30  -0.090   -1.2
TXN   TXN  -25    -50    X     25  -625 -0.50  -0.010   -1.2
DELL DELL   50      0    S     50 -5500 -0.15  -0.020   -1.2
GOOG GOOG    0     50    B     50  1000  1.10   0.010    1.2
PAYX PAYX -100    -50    C     50  1250  0.53  -0.001    1.2
SCHW SCHW    0    100    B    100  5000  1.20  -0.010    1.3
YHOO YHOO  -10      0    C     10   200  1.20  -0.002    1.4
GM     GM  100    200    B    100  3000  1.00   0.100    1.6
HAL   HAL  -75      0    C     75  1500  1.70   0.001    1.8
MSFT MSFT    0    100    B    100  6000  1.50  -0.020    1.9
\end{Soutput}
\end{Schunk}


\subsection{Calculating chunks}

Calculating chunks for a long-short portfolio functions in almost the
same manner as it would for a long-only portfolio.  We set the market
value of each chunk to be 2,000
in the call to \texttt{new}.


\begin{Schunk}
\begin{Sinput}
> sub.chunks
\end{Sinput}
\begin{Soutput}
       side rank.t chunk.shares chunk.mv tca.rank
AMD.1     S   -1.8           20    -2000     -1.8
AMD.2     S   -1.8           20    -2000     -1.8
AMD.3     S   -1.8           10    -1000     -1.8
APPL.1    X   -1.2           67    -2010     -1.2
APPL.2    X   -1.2            8     -240     -1.2
DELL.1    S   -1.2           18    -1980     -1.2
DELL.2    S   -1.2           18    -1980     -1.2
DELL.3    S   -1.2           14    -1540     -1.2
GM.1      B    1.6           67     2010      1.6
GM.2      B    1.6           33      990     -4.4
GOOG.1    B    1.2           50     1000      1.2
HAL.1     C    1.8           75     1500      1.8
HPQ.1     X   -1.7           50     -750     -1.7
MSFT.1    B    1.9           33     1980      1.9
MSFT.2    B    1.9           33     1980      1.9
MSFT.3    B    1.9           33     1980      1.9
MSFT.4    B    1.9            1       60      1.9
PAYX.1    C    1.2           50     1250      1.2
QCOM.1    S   -1.4           11    -2090     -1.4
QCOM.2    S   -1.4           11    -2090     -1.4
QCOM.3    S   -1.4            3     -570     -1.4
SCHW.1    B    1.3           40     2000      1.3
SCHW.2    B    1.3           40     2000      1.3
SCHW.3    B    1.3           20     1000      1.3
TXN.1     X   -1.2           25     -625     -1.2
YHOO.1    C    1.4           10      200      1.4
\end{Soutput}
\end{Schunk}

Aside from the addition of cover and short chunks, the chunk table
should appear exactly as it does in section \ref{calcChunks lo}.


\subsection{Calculating Swaps}

Swaps work slightly differently with a long-short tradelist than with
a long-only tradelist.  In a long-only tradelist we only have to pair
buys and sells, but in a long-short tradelist we have to pair buys,
sells, shorts, and covers.  The \texttt{calcSwaps} method accounts for
this by matching trades within a side.  We pair shorts with covers and
buys with sells:


\begin{Schunk}
\begin{Sinput}
> swaps.sub
\end{Sinput}
\begin{Soutput}
              side.enter tca.rank.enter side.exit tca.rank.exit rank.gain
MSFT.1,NA.0            B            1.9         S      -10000.0   10001.9
MSFT.2,NA.0            B            1.9         S      -10000.0   10001.9
MSFT.3,NA.0            B            1.9         S      -10000.0   10001.9
MSFT.4,NA.0            B            1.9         S      -10000.0   10001.9
HPQ.1,NA.0             X           -1.7         C       10000.0   10001.7
GM.1,NA.0              B            1.6         S      -10000.0   10001.6
APPL.1,NA.0            X           -1.2         C       10000.0   10001.2
APPL.2,NA.0            X           -1.2         C       10000.0   10001.2
TXN.1,NA.0             X           -1.2         C       10000.0   10001.2
SCHW.1,AMD.1           B            1.3         S          -1.8       3.1
SCHW.2,AMD.2           B            1.3         S          -1.8       3.1
SCHW.3,AMD.3           B            1.3         S          -1.8       3.1
GOOG.1,QCOM.1          B            1.2         S          -1.4       2.6
GM.2,QCOM.2            B           -4.4         S          -1.4      -3.0
NA.0,HAL.1             X        10000.0         C           1.8   -9998.2
NA.0,QCOM.3            B       -10000.0         S          -1.4   -9998.6
NA.0,YHOO.1            X        10000.0         C           1.4   -9998.6
NA.0,DELL.1            B       -10000.0         S          -1.2   -9998.8
NA.0,DELL.2            B       -10000.0         S          -1.2   -9998.8
NA.0,DELL.3            B       -10000.0         S          -1.2   -9998.8
NA.0,PAYX.1            X        10000.0         C           1.2   -9998.8
\end{Soutput}
\end{Schunk}

In the \texttt{side.enter} column we list buys (\texttt{B}) and shorts
(\texttt{X}) because the only way to enter a side is by initially
buying or shorting a stock.  Sells and covers move us closer to
exiting the position which is why we put these trades in the
\texttt{side.exit} column.  Like in previous examples, the labels
describe the swaps.  The value to the left of the comma is the name of
buy or short and the name to the right of the comma is the name of a
sell or cover.  The number following the period is the chunk number of
the stock involved in the trade.

Dummy chunks work similarly for long-short portfolios as they do for
long-only portfolios.  The main difference is that we must create
dummy shorts and covers to pair with real covers and shorts.  We
create 16 dummy chunks.
The dummy chunks at the head of the swaps table exist because the
current portfolio has a lesser market value than the target portfolio.
To increase the market value of the current portfolio we want to make
more buys and covers than sells.  The dummy chunks at the tail of the
table were created because we ran out of shorts and buys to match with
real covers and sells.  We assign this type of dummy trade a poor
trade-cost adjusted rank.

\subsection{The \texttt{calcSwapsActual} Method}

The \texttt{calcSwapsActual} method works in almost exactly the same
way as it does for a long-only tradelist.

\begin{Schunk}
\begin{Sinput}
> sub.swaps.actual <- tl@swaps.actual[, c("side.enter", "tca.rank.enter", 
+     "side.exit", "tca.rank.exit", "rank.gain")]
\end{Sinput}
\end{Schunk}

\begin{Schunk}
\begin{Sinput}
> sub.swaps.actual
\end{Sinput}
\begin{Soutput}
              side.enter tca.rank.enter side.exit tca.rank.exit rank.gain
MSFT.1,NA.0            B            1.9         S      -10000.0   10001.9
MSFT.2,NA.0            B            1.9         S      -10000.0   10001.9
MSFT.3,NA.0            B            1.9         S      -10000.0   10001.9
MSFT.4,NA.0            B            1.9         S      -10000.0   10001.9
HPQ.1,NA.0             X           -1.7         C       10000.0   10001.7
GM.1,NA.0              B            1.6         S      -10000.0   10001.6
APPL.1,NA.0            X           -1.2         C       10000.0   10001.2
APPL.2,NA.0            X           -1.2         C       10000.0   10001.2
TXN.1,NA.0             X           -1.2         C       10000.0   10001.2
SCHW.1,AMD.1           B            1.3         S          -1.8       3.1
SCHW.2,AMD.2           B            1.3         S          -1.8       3.1
SCHW.3,AMD.3           B            1.3         S          -1.8       3.1
GOOG.1,QCOM.1          B            1.2         S          -1.4       2.6
GM.2,QCOM.2            B           -4.4         S          -1.4      -3.0
\end{Soutput}
\end{Schunk}

We do not remove any swaps because we set the \texttt{turnover} equal
to the unsigned market value of the candidate trades.  If we
decrease \texttt{turnover}, some of the swaps will be excised.


\begin{Schunk}
\begin{Sinput}
> tl@turnover <- nt - tl@chunk.usd
\end{Sinput}
\end{Schunk}

We set turnover to equal the turnover necessary to complete all of the
candidate trades (\texttt{nt}), minus the maximum size of a chunk.
This guarantees that we do not make trade the worst swap, in this case
NA.0,PAYX.1.  By lowering
\texttt{turnover} we caused the worst ranked swap to be removed.



\begin{Schunk}
\begin{Sinput}
> sub.swaps.actual
\end{Sinput}
\begin{Soutput}
              side.enter tca.rank.enter side.exit tca.rank.exit rank.gain
MSFT.1,NA.0            B            1.9         S      -10000.0   10001.9
MSFT.2,NA.0            B            1.9         S      -10000.0   10001.9
MSFT.3,NA.0            B            1.9         S      -10000.0   10001.9
MSFT.4,NA.0            B            1.9         S      -10000.0   10001.9
HPQ.1,NA.0             X           -1.7         C       10000.0   10001.7
GM.1,NA.0              B            1.6         S      -10000.0   10001.6
APPL.1,NA.0            X           -1.2         C       10000.0   10001.2
APPL.2,NA.0            X           -1.2         C       10000.0   10001.2
TXN.1,NA.0             X           -1.2         C       10000.0   10001.2
SCHW.1,AMD.1           B            1.3         S          -1.8       3.1
SCHW.2,AMD.2           B            1.3         S          -1.8       3.1
SCHW.3,AMD.3           B            1.3         S          -1.8       3.1
GOOG.1,QCOM.1          B            1.2         S          -1.4       2.6
GM.2,QCOM.2            B           -4.4         S          -1.4      -3.0
\end{Soutput}
\end{Schunk}




\subsection{Calculating actual chunks}

The \texttt{calcchunksActual} method works similarly to the way it
does for a long-only tradelist:


\begin{Schunk}
\begin{Sinput}
> sub.chunks.actual
\end{Sinput}
\begin{Soutput}
       side alpha ret.1.d rank.t tca.rank chunk.shares chunk.mv chunk
MSFT.1    B  1.50  -0.020    1.9      1.9           33     1980     1
MSFT.2    B  1.50  -0.020    1.9      1.9           33     1980     2
MSFT.3    B  1.50  -0.020    1.9      1.9           33     1980     3
MSFT.4    B  1.50  -0.020    1.9      1.9            1       60     4
HPQ.1     X -1.30  -0.002   -1.7     -1.7           50     -750     1
GM.1      B  1.00   0.100    1.6      1.6           67     2010     1
APPL.1    X -0.30  -0.090   -1.2     -1.2           67    -2010     1
APPL.2    X -0.30  -0.090   -1.2     -1.2            8     -240     2
TXN.1     X -0.50  -0.010   -1.2     -1.2           25     -625     1
SCHW.1    B  1.20  -0.010    1.3      1.3           40     2000     1
SCHW.2    B  1.20  -0.010    1.3      1.3           40     2000     2
SCHW.3    B  1.20  -0.010    1.3      1.3           20     1000     3
GOOG.1    B  1.10   0.010    1.2      1.2           50     1000     1
GM.2      B  1.00   0.100    1.6     -4.4           33      990     2
AMD.1     S -0.94  -0.040   -1.8     -1.8           20    -2000     1
AMD.2     S -0.94  -0.040   -1.8     -1.8           20    -2000     2
AMD.3     S -0.94  -0.040   -1.8     -1.8           10    -1000     3
QCOM.1    S -0.36  -0.005   -1.4     -1.4           11    -2090     1
QCOM.2    S -0.36  -0.005   -1.4     -1.4           11    -2090     2
       dummy.quality
MSFT.1          <NA>
MSFT.2          <NA>
MSFT.3          <NA>
MSFT.4          <NA>
HPQ.1           <NA>
GM.1            <NA>
APPL.1          <NA>
APPL.2          <NA>
TXN.1           <NA>
SCHW.1          <NA>
SCHW.2          <NA>
SCHW.3          <NA>
GOOG.1          <NA>
GM.2            <NA>
AMD.1           <NA>
AMD.2           <NA>
AMD.3           <NA>
QCOM.1          <NA>
QCOM.2          <NA>
\end{Soutput}
\end{Schunk}

We have changed the swaps back into chunks.  The additional work for a
long-short portfolio involves converting buy/sell and short/cover
swaps into chunks instead of just dealing with buy/sell chunks.

\subsection{The \texttt{calcActual} Method}

The \texttt{calcActual} method works almost exactly the same way it
does for a long-only tradelist: 

\begin{Schunk}
\begin{Sinput}
> tl@actual
\end{Sinput}
\begin{Soutput}
       id side shares    mv alpha ret.1.d rank.t
AMD   AMD    S     50 -5000 -0.94  -0.040   -1.8
APPL APPL    X     75 -2250 -0.30  -0.090   -1.2
GM     GM    B    100  3000  1.00   0.100    1.6
GOOG GOOG    B     50  1000  1.10   0.010    1.2
HPQ   HPQ    X     50  -750 -1.30  -0.002   -1.7
MSFT MSFT    B    100  6000  1.50  -0.020    1.9
QCOM QCOM    S     22 -4180 -0.36  -0.005   -1.4
SCHW SCHW    B    100  5000  1.20  -0.010    1.3
TXN   TXN    X     25  -625 -0.50  -0.010   -1.2
\end{Soutput}
\end{Schunk}

We ``roll-up'' all the chunks into single orders.

\section{Conclusion}

With intelligently defined sorts, the \texttt{portfolio} package is a
powerful tool for managing equity portfolios.  Nonetheless, the
\texttt{tradelist} code could stand for improvement in certain areas,
particularly the area of trade-cost adjustment.  The current method of
using discrete and static boundaries for determining trade-adjusted
rank should be replaced by a trade-cost adjustment function.
Nonetheless, we believe that our package makes the difficult problem
of trading a little bit easier.

\end{document}

%% \subsection{Expressing preferences amongst trades}

%% We want to buy stocks that will increase in price.  Buy low, sell
%% high.  We maintain a list of stocks which we believe will increase in
%% price.  With each stock we associate a value, \emph{alpha}, which
%% predicts future changes in price.  

%% \begin{table}[!htbp]
%%   \begin{tabular}[c]{rr|rr}
%%     stock & alpha & stock & alpha \\
%%     \hline
%%     EBAY &  2.50 & AMD  &  -3.02  \\
%%     MSFT &  2.49 & AET  &  -2.84  \\
%%     SCHW &  2.12 & QCOM &  -2.20  \\
%%     GOOG &  1.85 & HAL  &         \\
%%     GM   &  1.57 & AAPL &         \\
%%     IBM  &  0.75 & HPQ  &         \\
%%     YHOO &  0.23 & SUNW &         \\
%%     \hline
%%   \end{tabular}
%% \end{table}

%% Table \ref{portfolio difference} shows our current portfolio, target
%% portfolio, and the diffence between the two.  

%% \begin{table}[!htbp]
%%   \begin{tabular}[c]{rrr|rrr|rr}

%%     stock & shares & price & stock & shares & price & stock & $\Delta$ shares \\
%%     \hline
%%     IBM   &  10  & 10 & IBM   &  20  & 10 & EBAY  & 10  \\
%%     GM    &  10  & 15 & GM    &  20  & 15 & GM    & 10  \\
%%     EBAY  &  10  & 20 & EBAY  &  20  & 20 & GOOG  & 10  \\
%%     GOOG  &  40  & 10 & GOOG  &  50  & 10 & IBM   & 10  \\
%%     MSFT  &  10  & 15 & MSFT  &  20  & 15 & MSFT  & 10  \\
%%     SCHW  &   0  & 20 & SCHW  &  15  & 20 & SCHW  & 15  \\
%%     \hline
%%   \end{tabular}
%%   \caption[portfolio difference]{Current portfolio, target portfolio,
%%   and portfolio difference (right, center, left).\label{portfolio
%%   difference}}
%% \end{table}
